% %%%%%%%%%%%%%%%%%%%%%%%%%%%%%%%%%%%%%%%%%%%%%%%%%%%%%%%%%%%%%%%%%%%%%%%%%%%%
\chapter{Introduction}%
\label{chap:intro}
% %%%%%%%%%%%%%%%%%%%%%%%%%%%%%%%%%%%%%%%%%%%%%%%%%%%%%%%%%%%%%%%%%%%%%%%%%%%%

Here goes the text. Use macros in \texttt{macros.tex}, such as
\XXX{highlighting draft comments}. \todo{Someone may like
\texttt{\\todo\{this\}}}. Or 
English\to{}Czech, or define yours \XXX{next time continue here.}

You can use glosses, such as \gls{mt} that generate the
expanded version with the abbreviation during the first use, and then only
the acronym for next time. \Gls{mt} shows the capitalized version. Define
the glosses in \texttt{acronyms.tex}.

Download and unzip the fresh ACL anthology.

Use \texttt{make} for compilation. See Makefile for some special useful commands, such as
\texttt{make watch}, \texttt{make normostrany}, etc.

Don't forget about the PDF/A compliance check. You can download veraPDF into
\texttt{vera} dir, and then \texttt{make vera}. If it prints FAIL, the problem will
be very likely in images included as pdf. See ÚFAL
wiki\furl{https://wiki.ufal.ms.mff.cuni.cz/internal:thesis} for help.
