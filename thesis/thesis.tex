%%% The main file. It contains definitions of basic parameters and includes all other parts.

%% Settings for single-side (simplex) printing
% Margins: left 40mm, right 25mm, top and bottom 25mm
% (but beware, LaTeX adds 1in implicitly)
\documentclass[12pt,notitlepage,a4paper,openright]{report}
\pagestyle{plain}

\PassOptionsToPackage{hyperfootnotes=false}{hyperref}

% fix pdfx
\usepackage{etoolbox}
% \makeatletter
% \@ifl@t@r\fmtversion{2021-06-01}%
%  {\AddToHook{package/after/xmpincl}
%    {\patchcmd\mcs@xmpincl@patchFile{\if\par}{\ifx\par}{}{\fail}}}{}
% \makeatother

\usepackage[usenames,dvipsnames,svgnames,table,rgb]{xcolor}
\usepackage[a-2u]{pdfx}
\usepackage{fontspec}
\usepackage[czech,english]{babel}
\usepackage{lmodern}
\usepackage{textcomp}
\usepackage[defaultlines=4,all]{nowidow}

% Turn this on when needed:
%\usepackage{microtype}

\usepackage{graphicx}
\usepackage[twoside, inner=3.7cm, outer=2.9cm, top=2.6cm, bottom=3.4cm]{geometry}
\usepackage{thesis}
\usepackage[round]{natbib}
\usepackage{multirow}
\usepackage{arydshln} % dashed lines in tables
\usepackage{array}
\usepackage{amssymb,latexsym,pifont}
\usepackage{amsmath}
\usepackage{enumitem} % custom lists
\usepackage[normalem]{ulem} % underlining
\usepackage{setspace} % line spacing
\usepackage{varioref} % nice references (above/below)
\usepackage[above,section]{placeins} % avoid figures pushed at end of chapters
\usepackage{listings}

\usepackage{tabularx}
\usepackage{booktabs} % nicer lines in table
\usepackage{multicol}
\usepackage{tikz}
\usepackage{pgfplots}
\pgfplotsset{compat=1.17}
\usepackage{gnuplot-lua-tikz}
\usetikzlibrary{shapes.geometric}
\usepackage{epstopdf}
\usepackage{algorithmicx}
\usepackage{algorithm}
\usepackage{algpseudocode}
\usepackage{mathtools}
\usepackage{quoting,xparse}

% acronyms and glossaries
\usepackage[acronym, nomain]{glossaries}
\usepackage[shortcuts=ac]{glossaries-extra}
\makeglossaries
\preto\chapter{\glsresetall}

\setabbreviationstyle[acronym]{long-short}

\usepackage{subcaption} % sub figures in a fiture
\usepackage{standalone} % include standoalone tikz images
\usepackage{bibentry}

% hack bibentry command for list of publications
\makeatletter
\renewcommand\bibentry[1]{\nocite{#1}{\frenchspacing
     \@nameuse{BR@r@#1\@extra@b@citeb}}}
\makeatother


\definecolor{mydarkblue}{rgb}{0,0.08,0.45}
\hypersetup{ %
  colorlinks=true,
  linkcolor=mydarkblue,
  citecolor=mydarkblue,
  filecolor=mydarkblue,
  urlcolor=mydarkblue,
}
\renewcommand{\sectionautorefname}{Section}
\renewcommand{\subsectionautorefname}{Section}
\renewcommand{\subsubsectionautorefname}{Section}

% \hypersetup{
%     colorlinks=false,
%     pdfborder={0 0 0},
%     unicode=true,
% }

\newcommand*\myglsentry[1]{%
  \protect\ifglsused{#1}{%
    \glsentryshort{#1}%
  }{%
    \glsentrylong{#1}%
  }%
}


%\newacronym{ai}{AI}{artificial intelligence}
%\newacronym{agi}{AGI}{artificial general intelligence}
%\newacronym{ar}{AR}{autoregressive}
%\newacronym{bleu}{BLEU}{bilingual evaluation understudy}
%\newacronym{bow}{BoW}{bag of words}
%\newacronym{bpe}{BPE}{byte-pair encoding}
%\newacronym{bptt}{BPTT}{backpropagation through time}
%\newacronym{ce}{CE}{cross-entropy}
%\newacronym{cf}{CF}{catastrophic forgetting}
%\newacronym{ci}{CI}{catastrophic interference}
%\newacronym{cifar}{CIFAR-10}{Canadian Institute For Advanced Research}
%\newacronym{cnn}{CNN}{convolutional neural network}
%\newacronym{comet}{COMET}{Crosslingual Optimized Metric for Evaluation of Translation}
%\newacronym{dan}{DAN}{deep averaging network}
%\newacronym{dl}{DL}{deep learning}
%\newacronym{elbo}{ELBO}{evidence lower bound}
%\newacronym{em}{EM}{expectation-maximization}
%\newacronym{eos}{EOS}{end-of-sequence}
%\newacronym{ewc}{EWC}{elastic weight consolidation}
%\newacronym{ffn}{FFN}{feed-forward network}
%\newacronym{fi}{FI}{Fisher information}
%\newacronym{fim}{FIM}{Fisher information matrix}
%\newacronym{glue}{GLUE}{General Language Understanding Evaluation}
%\newacronym{gru}{GRU}{gated recurrent unit}
%\newacronym{hmm}{HMM}{hidden Markov model}
%\newacronym{idf}{IDF}{inverse document frequency}
%\newacronym{iid}{IID}{independent and identically distributed}
%\newacronym{il}{IL}{incremental learning}
%\newacronym{iwslt}{IWSLT}{International Conference on Spoken Language Translation}
%\newacronym{kl}{KL}{Kullback–Leibler}
%\newacronym{ldd}{LDD}{long-distance dependencies}
%\newacronym{lm}{LM}{language model}
%\newacronym{lstm}{LSTM}{long short-term Memory}
%\newacronym{map}{MAP}{maximum a posteriori}
%\newacronym{mee}{MEE}{maximum entropy estimation}
%\newacronym{mffn}{MFFN}{masked feed-forward network}
%\newacronym{mha}{MHA}{multi-head attention}
%\newacronym{mmha}{MMHA}{masked multi-head attention}
%\newacronym{ml}{ML}{machine learning}
%\newacronym{mlm}{MLM}{masked language model}
%\newacronym{mlp}{MLP}{multi-layered perceptron}
%\newacronym{mmt}{MMT}{multimodal machine translation}
%\newacronym{mnist}{MNIST}{Modified National Institute of Standards and Technology}
%\newacronym{mrt}{MRT}{minimum risk training}
%\newacronym{mscoco}{MSCOCO}{Microsoft Common Objects in Context}
\newacronym{mt}{MT}{machine translation}
%\newacronym{mtl}{MTL}{multi-task learning}
%\newacronym{nar}{NAR}{non-autoregressive}
%\newacronym{nli}{NLI}{natural language inference}
%\newacronym{nll}{NLL}{negative log-likelihood}
\newacronym{nlp}{NLP}{natural language processing}
%\newacronym{moe}{MoE}{mixture-of-experts}
\newacronym{nmt}{NMT}{neural machine translation}
%\newacronym{nn}{NN}{neural network}
%\newacronym{oov}{OOV}{out-of-vocabulary}
\newacronym{pbmt}{PBMT}{phrase-based machine translation}
%\newacronym{pi}{PI}{path integral}
%\newacronym{qa}{QA}{question answering}
%\newacronym{rbf}{RBF}{radial basis function}
%\newacronym{relu}{ReLU}{rectified linear unit}
%\newacronym{rl}{RL}{reinforcement learning}
%\newacronym{rouge}{ROUGE}{Recall-Oriented Understudy for Gisting Evaluation}
%\newacronym{rnn}{RNN}{recurrent neural network}
%\newacronym{sgd}{SGD}{stochastic gradient descent}
%\newacronym{smt}{SMT}{statistical machine translation}
%\newacronym{sota}{SoTA}{state-of-the-art}
%\newacronym{squad}{SQuAD}{Stanford Question Answering Dataset}
%\newacronym{ste}{STE}{straight-through estimator}
%\newacronym{ted}{TED}{Technology, Entertainment and Design}
%\newacronym{ter}{TER}{Translation Error Rate}
%\newacronym{tfidf}{TF-IDF}{term frequency-inverse document frequency}
%\newacronym{wmt}{WMT}{Conference on Machine Translation}

%\newacronym{wngt}{WNGT}{Workshop on Neural Generation and Translation}
%\newacronym{xlm}{XLM}{cross-lingual language model}
%\newacronym{dcrf}{DCRF}{dynamic-transition \myglsentry{crf}}
%\newacronym{emodd}{EM+ODD}{\myglsentry{em} training + \myglsentry{odd}}
%\newacronym{nat}{NAT}{non-autoregressive \myglsentry{nmt}}
%\newacronym{hintnat}{Hint-NAT}{hint-based training for \myglsentry{nat}}
%\newacronym{jmnat}{JM-NAT}{jointly masked model for \myglsentry{nat}}
%\newacronym{natreg}{NAT-REG}{\myglsentry{nat} with auxiliary regularization}
\newacronym{asr}{ASR}{automatic speech recognition}
%\newacronym{axe}{AXE}{aligned cross-entropy}
%\newacronym{bon}{BoN}{bag-of-ngrams}
%\newacronym{chrf}{ChrF}{character F-score}
%\newacronym{cmlm}{CMLM}{conditional masked language model}
%\newacronym{crf}{CRF}{conditional random fields}
%\newacronym{ctc}{CTC}{connectionist temporal classification}
%\newacronym{disco}{DisCo}{disentangled context}
%\newacronym{glat}{GLAT}{Glancing Transformer}
%\newacronym{levt}{LevT}{Levenshtein Transformer}
%\newacronym{lpd}{LPD}{length parallel decoding}
%\newacronym{lpd}{lpd}{length parallel decoding}
%\newacronym{lt}{LT}{Latent Transformer}
%\newacronym{npd}{NPD}{noisy parallel decoding}
%\newacronym{oaxe}{OaXE}{order-agnostic cross-entropy}
%\newacronym{odd}{ODD}{optimal deduplicated decoding}
\newacronym{smt}{SMT}{statistical machine translation}
%\newacronym{san}{SAN}{self-attentive network}
%\newacronym{smart}{SMART}{semi-autoregressive training}
%\newacronym{tpu}{TPU}{tensor processing unit}


\newacronym{EUROSAI}{EUROSAI}{European Organisation of Supreme Audit Institutions}
\newacronym{UN}{UN}{United Nations}
\newacronym{EU}{EU}{European Union}
\newacronym{sst}{SST}{simultaneous speech translation}
\newacronym{st}{ST}{speech translation}
\newacronym{TRL}{TRL}{Technology Readiness Level}
\newacronym{wer}{WER}{word error rate}
\newacronym{cat}{CAT}{computer assisted translation}
\newacronym{cai}{CAI}{computer assisted interpreting}
\newacronym{vad}{VAD}{voice activity detection}
\newacronym{si}{SI}{simultaneous interpreting}
\newacronym{cr}{CR}{Continuous Rating}
\newacronym{qe}{QE}{quality estimation}
\newacronym{ne}{NE}{Normalized Erasure}
\newacronym{ELITR}{ELITR}{European Live Translator}


% Czech babel conflicts with cline, hacky fix (http://tex.stackexchange.com/questions/111999/slovak-and-czech-babel-gives-problems-with-cmidrule-and-cline):
% - basically disables hyphenation in tables, but it's not used anyway so it doesn't matter
\preto\tabular{\shorthandoff{-}}
\preto\tikzpicture{\shorthandoff{-}}
%
%
\hyphenation{%
da-ta-sets
da-ta-set
} % -- custom hyphenation

\setmainfont[Ligatures=Common]{Libertinus Serif}
% \setmainfont[Ligatures=Common]{Linux Libertine O}
\setsansfont[Scale=MatchLowercase]{DejaVu Sans}
\setmonofont[Scale=MatchLowercase]{DejaVu Sans Mono}


\NewDocumentCommand{\bywhom}{m}{% the Bourbaki trick
  {\nobreak\hfill\penalty50\hskip1em\null\nobreak
   \hfill\mbox{\normalfont(#1)}%
   \parfillskip=0pt \finalhyphendemerits=0 \par}%
}

\NewDocumentEnvironment{pquotation}{m}
  {\begin{quoting}[
     indentfirst=true,
     leftmargin=\parindent,
     rightmargin=\parindent]\itshape}
  {\bywhom{#1}\end{quoting}}

\setstretch{1.1} % line spacing

\expandafter\def\expandafter\quote\expandafter{\quote\small} % smaller quotations font


% orphan & widow control
%\clubpenalty 10000
%\widowpenalty 10000

% gaps between text and footnotes
\def\footnoteskip#1{
  \renewcommand\footnoterule{
     \vspace{#1}
     \hrule width 0.4\columnwidth%
     \vspace{3pt}
}
}
\footnoteskip{0.8em}


\setcounter{tocdepth}{2}
\setcounter{secnumdepth}{2}

%% cutting down warnings
%\hfuzz=2pt
%\hbadness=10000

% force-ordering citations according to dummy keys
\newcommand{\dummybiborderkey}[1]{}

\def\XXX#1{\textcolor{red}{XXX #1}}
%\def\todo#1{\textcolor{red}{TODO #1}}

% Dominik couldn't find a way to use clever references working, like
% \Cref{chap:intro} / \Cref{sec:method1} . So there's \Chapref and Sref.
\def\Chapref#1{Chapter~\ref{#1}}
\def\Sref#1{Section~\ref{#1}}
% but \Cref works for \Cref{tab:x} and \Cref{fig:y}



\def\cite#1{\citep{#1}} %   (Johnson et al., 2017)
% \citet{johnson-etal-2017-googles,dabre-survey} => Johnson et al. (2017)
% \citep{johnson-etal-2017-googles,dabre-survey} => (Johnson et al., 2017)
\def\inparcite#1{\citealp{#1}} % should be Smith, 2012


\def\furl#1{\footnote{\url{#1}}}
\def\hi#1{\textit{#1}}% highlight a term

\def\to{$\rightarrow$}
\def\red#1{\textcolor{red}{#1}}
\def\blue#1{\textcolor{blue}{#1}}


\newcommand{\veryshortarrow}[1][3pt]{\mathrel{%
     \vcenter{\hbox{\rule[-.5\fontdimen8\textfont3]{#1}{\fontdimen8\textfont3}}}%
     \mkern-4mu\hbox{\usefont{U}{lasy}{m}{n}\symbol{41}}}}

\newcommand{\paperdisclaim}[1]{%
\begin{center}\begin{minipage}{0.9\textwidth}
\footnotesize\it #1
\end{minipage}\end{center}
}

\def\ignorecolumn#1\unskip{}

\title{Data-to-Text Generation with Neural Language Models}
% \title{Techniques for Neural Data-to-Text Generation}

\def\fulldate{}
\author{Zdeněk Kasner}
\date{2024}
\dept{Institute of Formal and Applied Linguistics}
\supervisor{Mgr. et Mgr. Ondřej Dušek, Ph.D.}
\studyprogram{Computer Science}
\studyfield{Computational Linguistics}


\begin{document}

%
%
%
\renewcommand{\thepage}{\roman{page}}
\renewcommand\cite{\citep}
\selectlanguage{english}
\maketitle

\pagestyle{plain}
\normalsize
\setcounter{page}{2}

\cleardoublepage{}
\ \vspace{10mm}

\noindent \it

\vspace{\fill}
\noindent \rm
I declare that I carried out this doctoral thesis independently,
and only with the cited sources, literature and other professional sources.

I understand that my work relates to the rights and obligations
under the Act No.~121/2000 Coll., the Copyright Act, as amended,
in particular the fact that Charles University has the right
to conclude a license agreement on the use of this work as a school work
pursuant to Section~60 paragraph~1 of the Copyright Act.

\vspace{2cm}
\noindent Prague, \today \hspace{\fill}\theauthor % doplňte patřičné
% datum, jméno a
% příjmení

%%%   Do not forget to SIGN the printed book here!
%%%                  *********


\cleardoublepage{} % new page
\pagestyle{plain}

\addcontentsline{toc}{chapter}{English Abstract}

%\selectlanguage{english}
\begin{description}[leftmargin=7.5em,labelwidth=7em,labelindent=0em,labelsep=0.5em]
  \item[Title:] \thetitle{}
  \item[Author:] \theauthor{}
  \item[Department:] \thedept{}
  \item[Supervisor:] \thesupervisor{},\\ \thedept{}
\end{description}
\subsubsection{Abstract:}

Data-to-text generation systems need to produce texts with high levels of semantic accuracy. Rule-based systems can guarantee this aspect, but their fluency and adaptability to new domains remain limited. Meanwhile, neural language models can easily generate fluent texts and adapt to new domains but are notoriously prone to producing inaccurate outputs. This thesis explores how to efficiently employ neural components in data-to-text generation systems to get the best of both worlds. We focus on approaches based on pretrained transformer language models. Primarily, the models serve as building blocks for data-efficient and robust data-to-text generation systems. Along with that, we introduce model-based evaluation metrics, focusing on detecting errors in data-to-text outputs, and a toolkit for preprocessing and visualizing data-to-text generation datasets. We also analyze the behavior of pretrained and large language models in specific scenarios, including describing individual relations in knowledge graphs and generating texts from standard data formats. We conclude that while employing neural language models in data-to-text generation remains a delicate endeavor, neural components can improve the fluency of the output texts and make the systems adaptable to new domains. At the same time, the semantic accuracy of the outputs can remain high if the models are used for specific, well-defined subtasks for improving text quality. For future research, we emphasize the need for benchmarking with suitable evaluation metrics on real-world use cases.

\begin{description}[leftmargin=7.5em,labelwidth=7em,labelindent=0em,labelsep=0.5em]
  %
  \item[Keywords:] TODO
    %
\end{description}


\cleardoublepage{}
\addcontentsline{toc}{chapter}{Czech Abstract}
\selectlanguage{czech}
\begin{description}[leftmargin=7.5em,labelwidth=7em,labelindent=0em,labelsep=0.5em]
  \item[Název práce:] TODO
  \item[Autor:] \theauthor{}
  \item[Katedra:] Ústav formální a aplikované lingvistiky
  \item[Vedoucí práce:] \thesupervisor,\\ Ústav formální a aplikované lingvistiky
\end{description}

\subsubsection{Abstrakt:}

\XXX{Abstrakt je pouze v angličtině.}


\begin{description}[leftmargin=7.5em,labelwidth=7em,labelindent=0em,labelsep=0.5em]
  %
  \item[Klíčová slova:] TODO
    %
\end{description}

\selectlanguage{english}




\cleardoublepage{}
\ \vspace{10mm}

\addcontentsline{toc}{chapter}{Acknowledgements}
\subsection*{Acknowledgements}

{

  TODO
  % Here, you can thank anyone and say anything.

  %   \vspace{1\baselineskip}
  %   \noindent
  %   This is how I separated different kinds of thank-yous.

  %   \vspace{1\baselineskip}
  %   \noindent
  %   ... continued. 
}

\vfill


{\noindent\footnotesize %
  This work has been using language resources and tools developed and/or stored and/or distributed by the  LINDAT/CLARIN project of the Ministry of Education, Youth and Sports of the Czech Republic (project LM2015071).
}

\cleardoublepage{}
\addcontentsline{toc}{chapter}{Table of Contents}
\tableofcontents % automatically generated

\cleardoublepage{}
\renewcommand{\chapterheadstartvskip}{\vspace*{-10mm}} % chapter spacing
\setstretch{1.2} % line spacing

%
% TEXT START
%
\renewcommand{\thepage}{\arabic{page}}
\setcounter{page}{1}




\sloppy
% % %%%%%%%%%%%%%%%%%%%%%%%%%%%%%%%%%%%%%%%%%%%%%%%%%%%%%%%%%%%%%%%%%%%%%%%%%%%%
\chapter{Introduction}
\label{chap:intro}
Producing \emph{natural language} comes \emph{natural} to us, humans.
The key to computers' versatility and efficiency---their ``language''---are data structures: arrays and lists, trees and graphs, tables and databases.
Without appropriate tools, reading structured data is to most people like deciphering a foreign language. What is the best tool to undestand it? The problem lies not just in the unfamiliar format of such data, but in its scale. As the volume of structured data in our world is ever-growing, it becomes rather tempting to turn the question around: Can we instead make the computer translate the data to a language we already understand?

The attempts at generating natural language with a computer date back to the 1950s, when IBM researchers succesfully used a computer for translating between English and Russian \cite{sheridan1955research}. Shortly after, the work of \citet{chomsky1957syntactic} introduced formal grammar, providing a principled way for generating language with a set of rules. These initial successes stirred a lot of excitement; fully automated human-level language generation seemed within reach. In the 1960s, people slowly began to notice its difficulties: for example, \citet{yngve1961random} notes there is ``surprisingly wide linguistic diversity'' when constructing grammar rules for the first ten sentences of a children's book. Still, the field of text generation gained momentum and descriptions of text generation systems started to appear \cite[\emph{inter alia}]{woolley-1969-automatic,meehan-1975-using,mcdonald-1975-framework,wang-1980-computational}. The report on the state of the art in text generation in 1982 predicted that within five years:
\begin{pquotation}{\citealp{mann-1982-text}}
    The resulting system can be expected to create acceptable, effective texts, limited by quality considerations to be about one page in length.\hspace{2cm}
\end{pquotation}
% The seminal work of \citet{winograd1971procedures} on SHRDLU, a system able to manipulate a block world according to user instructions, only glosses over presenting the state of the world to the user:
% \begin{pquotation}{\citealp[p.384]{winograd1971procedures}}
%     Responses can be made as complex and varied as we want, since they are created by the programmer, and the program only repeats them.
% \end{pquotation}
% In other words, spending enough time with programming appropriate rules was considered enough for automating language generation.

Fast forward to the present, and the research community is beaming with excitement again, this time about the unprecedented capabilities of neural \aclp{lm} (\acsp{lm}) in generating fluent texts \cite{radford2019language,brown2020language}. In the end, it took us over fifty years to build such systems. Similarly to other tasks in \ac{ai}, from object recognition \cite{papert1966summer} to self-driving cars \cite{autonomouscars}, the apparent ease of the task for humans has proven deceptive.  To achieve progress, we had to move away from linguistic theories and rule-based systems, re-defining our systems in terms of data-based approaches and generic learning algorithms.

% In the preceding decades, automatic \ac{nlg} systems were built using rules and grammars \cite{reiterBuildingAppliedNatural1997,gattSurveyStateArt2018}. 
\Ac{nlg} has meanwhile established itself as a standalone scientific discipline, with its journals, conferences, and stable base of researchers.\footnote{See the history of SIGGEN meetings: \url{https://aclanthology.org/sigs/siggen/}.} The research in the preceding decades was characterized by using a varied assortment of tools including grammars, formalisms, linguistic theories, and custom components. Combining these tools was understood as necessary for building text generation systems \cite{mann-1982-text,reiterBuildingAppliedNatural1997}. As a result, many systems from that time---from chart captioning systems \cite{mittalDescribingComplexCharts1998} and graph descriptors \cite{sunDomainIndependentSentence2006}, to weather forecast systems \cite{belzAutomaticGenerationWeather2008} and healthcare report generators \cite{portetAutomaticGenerationTextual2009}---were accurate and reliable, but domain-specific and rigid.


With neural models, \ac{nlp} as a research field, along with \ac{nlg} as one of its subfields, has changed dramatically \cite{gururaja2023build,li2023defining}. Most notably, these fields have become more experimental. While neural \acp{lm} opened up fascinating possibilities in building end-to-end systems and solving the long-standing issues with fluency and domain-independence \cite{ferreiraNeuralDatatotextGeneration2019,dusekEvaluatingStateoftheartEndtoEnd2020,sharmaInnovationsNeuralDatatotext2022}, working with neural models turned out to be closer to behavioral sciences than engineering \cite{holtzmanGenerativeModelsComplex2023}. As the researchers began to ``throw'' neural \acp{lm} at all sorts of problems, the issues concerning experimental design and evaluation came to the surface \cite{gehrmannRepairingCrackedFoundation2022}. Due to this, some researchers perceived the change as questionable at the very least \cite{reiter2020academic,gururaja2023build,michael2023nlp}. The shift towards experimental approaches has also created a gap between research and industry; the industry opted for established approaches meeting industrial standards instead of trying new research artifacts \cite{daleNaturalLanguageGeneration2020,daleNavigatingTextGeneration2023}.


Nevertheless, the progressive approach adopted by \ac{nlp} over the past few years turned out to have its merits. The general emphasis on openness, inherited from the \ac{ml} field---where publicly releasing papers, code, and models has become commonplace---has allowed everybody to stand on the proverbial shoulders of giants. Thanks to open-science initiatives such as arXiv.org\footnote{\url{https://arxiv.org/}} or HuggingFace\footnote{\url{https://huggingface.co/}}, research became more accessible to both researchers and the general public. The convergence towards generic approaches has also led to heavy cross-pollination of ideas, making specific solutions easily applicable to other tasks. As such, \ac{nlg} is helping to advance other areas of \ac{nlp} and contribute to general knowledge of the natural language, its production and processing.

Finally, as we gained ways to generate language that do not require starting from structured representations (summarize and paraphrase texts, continue text segments, generate stories and answers to questions, or describe images and videos), % \cite{Dong2021ASO}
the original field concerned with generating descriptions of structured data has adopted the---perhaps more apt---name of \emph{\ac{d2t} generation}.

This thesis tells a story about how \acl{d2t} generation and neural \aclp{lm} came together. On the way, it touches various facets of \ac{d2t} generation: from improving generation in a low-resource setting (\autoref{chap:low-res}), over evaluating generated texts (\autoref{chap:evaluation}), processing and visualizing data (\autoref{chap:tabgenie}), to interpreting system behavior (\autoref{chap:investigating}).
% After giving an overview of the field in the background section (\autoref{chap:background}), the thesis delves into specific experiments, but it can hopefully offer pointers to newcomers in the field along with a handful of fruitful ideas. 
The thesis inevitably reflects the shifts in \ac{nlp} between 2020 and 2024: from the preliminary attempts at generating fluent language with small pretrained \acp{lm}, all the way up to dealing with the hype surrounding the \acp{llm}. The approaches presented in this thesis are primarily motivated by the idea that adopting neural models in \ac{d2t} may help us solve some long-standing issues with flexibility and text fluency, which were out of reach for the best approaches from previous decades.




\section{Motivation}
\label{sec:rq}

The main goal of the thesis is to close the gap outlined in the introduction: turning experimental approaches into reliable and accurate \ac{d2t} generation systems. As a premise, we consider neural \acp{lm}\footnote{For brevity, we will commonly use ``\acp{lm}'' to denote ``neural \acp{lm}'' throughout this work unless stated otherwise.} as a useful tool of producing fluent and natural-sounding text.
% , with the potential to make a difference in \ac{d2t} systems compared to prior approaches. 
However, we do not take neural \acp{lm} as a one-size-fits-all solution. Instead, we carefully study how to integrate \acp{lm} in \ac{d2t} systems while following the strict demands on fluency, controllability, and semantic accuracy of the output.

The side goal of the thesis is then to \textit{understand}: understand the data we are dealing with, the outputs we can reasonably expect, and the behavior of neural-based systems in certain conditions. \ac{d2t} generation has several specifics that make it a good subject for this kind of study: its resource scarcity (due to which there are still questions that cannot be answered by scaling up the models), the tension between the established rule-based and new-coming neural approaches, and the fact that the specific format and size of the data makes it less suitable for end-to-end solutions.

To make the goals more tangible, we split them into the following research questions, which we address further on in the thesis:

\begin{description}
    \item[RQ1\label{rq:1}] \textbf{In which scenarios are \acp{lm} useful for \ac{d2t} generation?} First, it is crucial to identify the strong sides of \acp{lm} and get an intuition of where the models can make the most impact. How far can we get with \ac{lm}-only baselines? And are there outcomes that we can get with \acp{lm} that are better than previous approaches?
    \item[RQ2\label{rq:2}] \textbf{How to efficiently process the structured data with \acp{lm}?} With structured data, we need to deal with the fact that \acp{lm} were pre-trained on modeling plain text only. To efficiently leverage the knowledge in \acp{lm}---especially in low-resource settings---we need to find the way to transform the data into a suitable input format while keeping its structure (along with other information in the data) intact.
    \item[RQ3\label{rq:3}] \textbf{How to make \ac{lm}-based systems more controllable?} A neural component introduced in the \ac{d2t} generation system will inevitably make the system less controllable. The question is if we can minimize these issues by building systems out of smaller and simpler components, training the models for more predictable tasks, or producing intermediate outputs that can be manually examined.
    \item[RQ4\label{rq:4}] \textbf{How to evaluate the outputs of \ac{d2t} generation systems?} Evaluating generated text gets harder as the quality of the texts starts to approach the human level. Since human evaluation is costly and time-consuming, we study how to build automatic metrics that can be used for system development and evaluation. Particularly, we focus on the most pressing issue in \ac{d2t} generation: semantic accuracy of the generated texts with respect to the input data.
    \item[RQ5\label{rq:5}] \textbf{Do \ac{d2t} generation systems generalize to unseen domains and datasets?} \ac{d2t} generation systems are often evaluated on a limited subset of domains and datasets. Investigating how the models perform on unseen domains, multiple datasets, and real-world data would give us better picture of the limitations of the current approaches.

        % Understanding the abilities and limitations of the systems is crucial for further progress in the field. How does the format of the data influence the outputs of the models? Are \acp{lm} robust enough to replace rule-based approaches? And what are the most important problems to tackle in neural-based \ac{d2t} systems?
\end{description}



\section{Main Contributions}
\label{sec:contributions}


The following are our main contributions, following the research questions outlined above:
% \setlist[description]{labelindent=\parindent, leftmargin=0em}

\begin{description}[leftmargin=\widthof{\textbf{Ad RQ1\ \ }}]
    \item[Ad \ref{rq:1}] We show that  with a very \textbf{simple \ac{lm}-based finetuned baseline}, we can achieve strong results on a shared task of generating texts from a knowledge graph (\autoref{sec:finetuning}). We also point out the advantages and limitations of open \acp{llm} on \ac{d2t} generation in zero-shot settings (\autoref{sec:quintd}).
    \item[Ad \ref{rq:2}] We show how to \textbf{transform the data to intermediate text-like input} suitable for \acp{lm} using handcrafted or automatically extracted templates (\Cref{sec:iterative,sec:pipeline,sec:sem-acc}), rule-based \ac{nlg} methods (\autoref{sec:tok-eval}), and specialized \acp{lm} (\autoref{sec:rel2text}). We show that these methods can serve as a basis both for competitive neural-based \ac{d2t} generation systems and for novel \ac{lm}-based evaluation metrics.
    \item[Ad \ref{rq:3}] We show how we can limit \acp{lm} to the task of improving text fluency and use these \acp{lm} for building \textbf{more controllable \ac{d2t} generation systems} with an iterative approach (\autoref{sec:iterative}) and modular architecture (\autoref{sec:pipeline}). We show that these systems open up a new way of thinking about neural-based \ac{lm} with a different set of trade-offs than rule-based or end-to-end systems.
    \item[Ad \ref{rq:4}] We develop \textbf{\ac{lm}-based automatic metrics} for evaluating outputs of \ac{d2t} generation systems on the level of data item mentions (\autoref{sec:sem-acc}) and output tokens (\autoref{sec:tok-eval}). We show that the metrics achieve strong correlations with human judgment in comparison with other metrics.
    \item[Ad \ref{rq:5}] We \textbf{unify the format} of multiple \ac{d2t} generation datasets for easier processing and visualization (\autoref{sec:tabgenie}). Using novel datasets, we also \textbf{evaluate the output quality and semantic accuracy} of \acp{lm} across multiple \ac{d2t} tasks, data formats, and domains (\Cref{sec:quintd,sec:rel2text}).
\end{description}



\section{Thesis Overview}
\label{sec:overview}

The thesis is organized into the background chapter (\autoref{chap:background}), the content chapters (\Cref{chap:low-res,chap:evaluation,chap:tabgenie,chap:investigating}), and the concluding chapter (\autoref{chap:conclusions}).

The \Cref{chap:low-res,chap:evaluation,chap:tabgenie,chap:investigating}, which describe our contributions, are outlined in \autoref{tab:overview}. First, we describe our work on improving \ac{d2t} generation in low-resource scenarios in \autoref{chap:low-res}. We continue with our work on evaluating the semantic accuracy of \ac{d2t} generation in \autoref{chap:evaluation}. In \autoref{chap:tabgenie}, we describe \textsc{TabGenie}, our toolkit for processing and visualization of \ac{d2t} generation datasets. Finally, in \autoref{chap:investigating}, we present our experiments with cross-domain performance of pretrained and large \acp{lm} on \ac{d2t} generation.

\begin{table*}[t]
    \small
    \begin{tabular}{p{0.7cm}p{8.3cm}p{4cm}}
        \toprule
        \textbf{Sec.}         & \textbf{Topic}                                         & \textbf{Publication}                             \\ \midrule
        \multicolumn{3}{l}{\textbf{\autoref{chap:low-res}: Low-Resource Data-to-Text Generation}}                                         \\
        §\ref{sec:finetuning} & \ac{d2t} generation with a finetuned \ac{lm}           & \citet{kasnerTrainHardFinetune2020}              \\
        §\ref{sec:iterative}  & \ac{d2t} generation with an editing  \ac{lm}           & \citet{kasnerDatatoTextGenerationIterative2020}  \\
        §\ref{sec:pipeline}   & \ac{d2t} generation with a pipeline of \acp{lm}        & \citet{kasner2022neural}                         \\ \cdashlinelr{1-3}
        \multicolumn{3}{l}{\textbf{\autoref{chap:evaluation}: Evaluating Generated Texts}}                                                \\
        §\ref{sec:sem-acc}    & Evaluating \ac{d2t} with natural language inference    & \citet{dusekEvaluatingSemanticAccuracy2020}      \\
        §\ref{sec:tok-eval}   & Evaluating token-level accuracy of complex \ac{d2t}    & \citet{kasnerTextinContextTokenLevelError2021}   \\ \cdashlinelr{1-3}
        \multicolumn{3}{l}{\textbf{\autoref{chap:tabgenie}: Data Processing and Visualization}}                                           \\
        §\ref{sec:tabgenie}   & \textsc{TabGenie} toolkit for \ac{d2t} datasets        & \citet{kasnerTabGenieToolkitTabletoText2023}     \\ \cdashlinelr{1-3}
        \multicolumn{3}{l}{\textbf{\autoref{chap:investigating}: Investigating Domain Generalization}}                                    \\
        §\ref{sec:rel2text}   & Describing unseen triples in a knowledge graph         & \citet{kasnerMindLabelsDescribing2022}           \\
        §\ref{sec:quintd}     & \ac{d2t} generation across domains with open \acp{llm} & \citet{kasnerReferenceBasedMetricsAnalyzing2024} \\\bottomrule
    \end{tabular}

    \caption{Overview of the thesis.}
    \label{tab:overview}
\end{table*}

\paragraph{Publications} The thesis includes the content of eight publications written by the author of the thesis. Except for the paper \citet{dusekEvaluatingSemanticAccuracy2020}, where the experimental part was done by the author's supervisor, the author of the thesis was the main author of all the publications and executed major part of the work.\footnote{The contributions for publications with multiple authors are detailed in the respective chapters.} All the publications were (or are to be) published at top-tier \ac{nlp} conferences ACL, EACL and INLG.


% \include{02-method}
% % %%%%%%%%%%%%%%%%%%%%%%%%%%%%%%%%%%%%%%%%%%%%%%%%%%%%%%%%%%%%%%%%%%%%%%%%%%%%
\chapter{Conclusions}%
\label{chap:conslusions}
% %%%%%%%%%%%%%%%%%%%%%%%%%%%%%%%%%%%%%%%%%%%%%%%%%%%%%%%%%%%%%%%%%%%%%%%%%%%%

\chapter{Introduction}
\label{chap:intro}
Producing \textit{natural language} is \textit{natural} only to humans, to computers less so. The key to computers' versatility and efficiency -- their ``language'' -- are data structures: arrays, lists, trees and graphs, tables and databases. Once the data stored in these structures meets human eyes, it is vital that humans are able to understand and interpret it. Do we need domain experts to scrutinize over these structures, or can we hope to teach the computers to translate the data into our language?

The answer seems to be in favor of the latter. After all, the first attempts at producing language with computers date back to the dawn of computing, with audacious attempts of \textit{translating} between English and Russian in 1950's \cite{sheridan1955research}. With all the initial excitement, \textit{generating} English-only sentences seemed like a simpler task. Although in 1960's, people slowly began to ponder on the difficulties of generating language---\citet{yngve1961random} notes that creating grammar rules for a simple children's book is difficult since even the first ten sentences provide \textit{``surprisingly wide linguistic diversity''}---the overall sentiment was that language generation will soon be solved up to some minor details. The 461-page seminal work of \citet{winograd1971procedures} on SHRDLU, a system which can manipulate blocks according to user instructions, only glosses over presenting the state of the block world to the user:
\begin{pquotation}{\citealp[p.384]{winograd1971procedures}}
  [R]esponses can be made as complex and varied as we want, since they are created by the programmer, and the program only repeats them.
\end{pquotation}
Skipping ahead 50 years of research (to which we will return in detail in \autoref{sec:rule-d2t}), the research world beams with excitement again: neural \acp{lm} are now able to generate the long-sought \textit{complex} and \textit{varied} language. Similarly to other ideas in artificial intelligence---from object recognition \cite{papert1966summer} to self-driving cars \cite{autonomouscars}---the apparent ease of the task for humans has proven deceptive. To make progress, we needed to shift our attention linguistic theories and rule-based systems, re-defining our approach in terms of data-based approaches and generic learning algorithms.

Over the past decades, the goal of most of the \ac{nlg} systems turned out to be rather pragmatic: taking data from a particular system  and presenting them to the users in the form they will understand. The works in \ac{nlg} were the works in \textit{engineering}, striving to craft systems using well-understood tools to achieve particular results. Natural language, as one of the suitable mediums to describe the data in the computer, was used to present the data from healthcare, to weather forecasts and sport reports. The \ac{nlg} systems built until the arrival of neural models were accurate and reliable, if only a bit too domain-specific and rigid \cite{reiterBuildingAppliedNatural1997,gattSurveyStateArt2018}. The essence of these systems was transforming a non-linguistic inputs to linguistic outputs according to a sequence of rule. In some sense, we can therefore say that transforming data to text---what is now specifically called \ac{d2t} generation---was all there was to \ac{nlg}.

With neural models, the approaches to \glsxtrfull{d2t} generation are getting more experimental and help to advance the \glsxtrfull{nlp} in general.


% Data-to-text generation is the process of starting with the textual representation understandable to computers and ending up the the  textual representation undestandable to humans.

\section{Research Questions}
\label{sec:rq}
\section{Main Contributions}
\label{sec:contributions}
\section{Thesis Overview}
\label{sec:overview}

\chapter{Background}
\label{chap:background}
\section{Neural Language Models}
\label{sec:lms}
\subsection{Neural Networks}
\label{sec:nns}
\subsection{Transformer Architecture}
\label{sec:transformer}
\subsection{Pretrained Language Models}
\label{sec:plms}
\subsection{Large Language Models}
\label{sec:llms}
\section{Data-to-Text Generation}
\label{sec:d2t}
\subsection{Rule-based Approaches}
\label{sec:rule-d2t}
\subsection{Neural Approaches}
\label{sec:neural-d2t}
\subsection{Datasets}
\label{sec:datasets}
\subsection{Evaluation Metrics}
\label{sec:evaluation}

\chapter{Low-Resource Data-to-Text Generation}
\label{chap:low-res}
\section{Motivation}
\label{sec:low-res-mot}
\section{Finetuning LMs}
\label{sec:finetuning}
\subsection{WebNLG+ Shared Task}
\label{sec:webnlgp}
\subsection{Our Submission}
\label{sec:mbart}
\section{Iterative Template Fusion with Text-Editing LMs}
\label{sec:iterative}
\subsection{Text-Editing LMs}
\label{sec:text-editing}
\subsection{Experiments}
\label{sec:text-editing-exp}
\section{Pipelined Text-Based Operations with Pretrained LMs}
\label{sec:pipeline}
\subsection{Pipeline Operations}
\label{sec:pipeline-ops}
\subsection{Experiments}
\label{sec:pipeline-exp}

\chapter{Evaluating Generated Text}
\label{chap:evaluation}
\section{Motivation}
\label{sec:evalution-mot}
\section{Evaluating Semantic Accuracy}
\label{sec:sem-acc}
\subsection{Experiments}
\label{sec:sem-acc-exp}
\section{Token-Level Error Detection}
\label{sec:eval-token}
\subsection{Shared Task}
\label{sec:eval-st}
\subsection{Our Submission}
\label{sec:eval-ours}

\chapter{Data Processing and Visualization}
\label{chap:data}
\section{Motivation}
\label{sec:data-mot}
\section{TabGenie Toolkit}
\label{sec:tabgenie}
\subsection{Data Processing}
\label{sec:tabgenie-data}
\subsection{Web Interface}
\label{sec:tabgenie-web}
\subsection{Programming Interface}
\label{sec:tabgenie-cli}


\chapter{Investigating Model Capabilities}
\label{chap:investigating}
\section{Motivation}
\label{sec:investigating-mot}
\section{Describing Triples in Knowledge Graphs}
\label{sec:describing}
\subsection{Knowledge Graphs}
\label{sec:kgs}
\subsection{Rel2Text Dataset}
\label{sec:rel2text}
\subsection{Experiments}
\label{sec:rel2text-exp}
\section{Prompting Open LLMs}
\label{sec:prompting}
\subsection{\textsc{Quintd} Toolkit}
\label{sec:quintd}
\subsection{Experiments}
\label{sec:quintd-exp}




\chapter{Conclusions}
\label{chap:conclusions}


%
% TEXT END
%

\renewcommand{\chapterheadstartvskip}{\vspace*{0mm}} % chapter spacing

\cleardoublepage{}
\bibliographystyle{csplainnat}
\addcontentsline{toc}{chapter}{Bibliography}
{\small \bibliography{references}}

\cleardoublepage{}
\addcontentsline{toc}{chapter}{List of Abbreviations}
\renewcommand*{\acronymname}{List of Abbreviations}
\printglossary[type=\acronymtype,style=index]

\addcontentsline{toc}{chapter}{List of Tables}
{\small \listoftables\par}

\addcontentsline{toc}{chapter}{List of Figures}
{\small \listoffigures\par}

\cleardoublepage{}
\addcontentsline{toc}{chapter}{List of Publications}
\chapter*{List of Publications}

\phantom{\nobibliography*{references}}

% -----------------------------------------------------------------------------

\noindent\bibentry{kasnerTrainHardFinetune2020}
\begin{itemize}[noitemsep,topsep=0pt]

    \item Citations (without self-citations): 9
\end{itemize}\vspace{.5\baselineskip}

% -----------------------------------------------------------------------------
\noindent\bibentry{kasnerDatatoTextGenerationIterative2020}
\begin{itemize}[noitemsep,topsep=0pt]
    \item Citations (without self-citations): 17

\end{itemize}\vspace{.5\baselineskip}

% -----------------------------------------------------------------------------
\noindent\bibentry{kasner2022neural}
\begin{itemize}[noitemsep,topsep=0pt]
    \item Citations (without self-citations): 21

\end{itemize}\vspace{.5\baselineskip}

% -----------------------------------------------------------------------------
\noindent\bibentry{dusekEvaluatingSemanticAccuracy2020}
\begin{itemize}[noitemsep,topsep=0pt]
    \item Citations (without self-citations): 47

\end{itemize}\vspace{.5\baselineskip}


% -----------------------------------------------------------------------------
\noindent\bibentry{kasnerTextinContextTokenLevelError2021}
\begin{itemize}[noitemsep,topsep=0pt]
    \item Citations (without self-citations): 6

\end{itemize}\vspace{.5\baselineskip}


% -----------------------------------------------------------------------------
\noindent\bibentry{kasnerTabGenieToolkitTabletoText2023}
\begin{itemize}[noitemsep,topsep=0pt]
    \item Citations (without self-citations): 2

\end{itemize}\vspace{.5\baselineskip}


% -----------------------------------------------------------------------------
\noindent\bibentry{kasnerMindLabelsDescribing2022}
\begin{itemize}[noitemsep,topsep=0pt]

    \item The analysis of verbalizing relations in knowledge graphs with pretrained language models (\autoref{sec:rel2text}).
    \item Citations (without self-citations): 3

\end{itemize}\vspace{.5\baselineskip}


% -----------------------------------------------------------------------------
\noindent\bibentry{kasnerReferenceBasedMetricsAnalyzing2024}
\begin{itemize}[noitemsep,topsep=0pt]

    \item The analysis of data-to-text generation with open large language models (\autoref{sec:quintd}).
    \item Citations (without self-citations): 2

\end{itemize}\vspace{.5\baselineskip}

\vfill

\noindent Only publications relevant to this thesis are included. The number of
citations was computed using Semantic Scholar API. Total number of citations of
publications related to the topic of the thesis (without self-citations) by June 14, 2024: {\bf 107}.


\end{document}
