%%% The main file. It contains definitions of basic parameters and includes all other parts.

%% Settings for single-side (simplex) printing
% Margins: left 40mm, right 25mm, top and bottom 25mm
% (but beware, LaTeX adds 1in implicitly)
\documentclass[12pt,notitlepage,a4paper,openright]{report}
\pagestyle{plain}

\PassOptionsToPackage{hyperfootnotes=false}{hyperref}

% fix pdfx
\usepackage{etoolbox}
% \makeatletter
% \@ifl@t@r\fmtversion{2021-06-01}%
%  {\AddToHook{package/after/xmpincl}
%    {\patchcmd\mcs@xmpincl@patchFile{\if\par}{\ifx\par}{}{\fail}}}{}
% \makeatother

\usepackage[usenames,dvipsnames,svgnames,table,rgb]{xcolor}
\usepackage[a-2u]{pdfx}
\usepackage{fontspec}
\usepackage[czech,english]{babel}
\usepackage{lmodern}
\usepackage{textcomp}
\usepackage[defaultlines=4,all]{nowidow}

% Turn this on when needed:
%\usepackage{microtype}

\usepackage{graphicx}
\usepackage[twoside, inner=3.7cm, outer=2.9cm, top=2.6cm, bottom=3.4cm]{geometry}
\usepackage{thesis}
\usepackage[round]{natbib}
\usepackage{multirow}
\usepackage{arydshln} % dashed lines in tables
\usepackage{array}
\usepackage{amssymb,latexsym,pifont}
\usepackage{amsmath}
\usepackage{enumitem} % custom lists
\usepackage[normalem]{ulem} % underlining
\usepackage{setspace} % line spacing
\usepackage{varioref} % nice references (above/below)
\usepackage[above,section]{placeins} % avoid figures pushed at end of chapters
\usepackage{listings}

\usepackage{tabularx}
\usepackage{booktabs} % nicer lines in table
\usepackage{multicol}
\usepackage{tikz}
\usepackage{pgfplots}
\pgfplotsset{compat=1.17}
\usepackage{gnuplot-lua-tikz}
\usetikzlibrary{shapes.geometric}
\usepackage{epstopdf}
\usepackage{algorithmicx}
\usepackage{algorithm}
\usepackage{algpseudocode}
\usepackage{mathtools}

% acronyms and glossaries
\usepackage[acronym, nomain]{glossaries}
\usepackage[shortcuts=ac]{glossaries-extra}
\makeglossaries
\preto\chapter{\glsresetall}

\setabbreviationstyle[acronym]{long-short}

\usepackage{subcaption} % sub figures in a fiture
\usepackage{standalone} % include standoalone tikz images
\usepackage{bibentry}

% hack bibentry command for list of publications
\makeatletter
\renewcommand\bibentry[1]{\nocite{#1}{\frenchspacing
     \@nameuse{BR@r@#1\@extra@b@citeb}}}
\makeatother


\definecolor{mydarkblue}{rgb}{0,0.08,0.45}
\hypersetup{ %
  colorlinks=true,
  linkcolor=mydarkblue,
  citecolor=mydarkblue,
  filecolor=mydarkblue,
  urlcolor=mydarkblue,
}

% \hypersetup{
%     colorlinks=false,
%     pdfborder={0 0 0},
%     unicode=true,
% }

\newcommand*\myglsentry[1]{%
  \protect\ifglsused{#1}{%
    \glsentryshort{#1}%
  }{%
    \glsentrylong{#1}%
  }%
}


\newacronym{ai}{AI}{artificial intelligence}
\newacronym{bert}{BERT}{bidirectional encoder representations from Transformers}
\newacronym{bleu}{BLEU}{bilingual evaluation understudy}
\newacronym{bpe}{BPE}{byte-pair encoding}
\newacronym{chrf}{ChrF}{character F-score}
\newacronym{cnn}{CNN}{convolutional neural network}
\newacronym{d2t}{D2T}{data-to-text}
\newacronym{gru}{GRU}{Gated Recurrent Unit}
\newacronym{lm}{LM}{language model}
\newacronym{lstm}{LSTM}{Long Short-Term Memory}
\newacronym{ml}{ML}{machine learning}
\newacronym{mlm}{MLM}{masked language model}
\newacronym{mt}{MT}{machine translation}
\newacronym{nar}{NAR}{non-autoregressive}
\newacronym{nlp}{NLP}{natural language processing}
\newacronym{nlg}{NLG}{natural language generation}
\newacronym{nn}{NN}{neural network}
\newacronym{relu}{ReLU}{Rectified Linear Unit}
\newacronym{rnn}{RNN}{recurrent neural network}
\newacronym{sgd}{SGD}{stochastic gradient descent}
\newacronym{tpu}{TPU}{tensor processing unit}

% \newacronym{dcrf}{DCRF}{dynamic-transition \myglsentry{crf}}
% \newacronym{emodd}{EM+ODD}{\myglsentry{em} training + \myglsentry{odd}}
% \newacronym{nat}{NAT}{non-autoregressive \myglsentry{nmt}}

% \newacronym{hintnat}{Hint-NAT}{hint-based training for \myglsentry{nat}}
% \newacronym{jmnat}{JM-NAT}{jointly masked model for \myglsentry{nat}}
% \newacronym{natreg}{NAT-REG}{\myglsentry{nat} with auxiliary regularization}


% Czech babel conflicts with cline, hacky fix (http://tex.stackexchange.com/questions/111999/slovak-and-czech-babel-gives-problems-with-cmidrule-and-cline):
% - basically disables hyphenation in tables, but it's not used anyway so it doesn't matter
\preto\tabular{\shorthandoff{-}}
\preto\tikzpicture{\shorthandoff{-}}
%
%
\hyphenation{%
da-ta-sets
da-ta-set
} % -- custom hyphenation

\setmainfont[Ligatures=Common]{Libertinus Serif}
% \setmainfont[Ligatures=Common]{Linux Libertine O}
\setsansfont[Scale=MatchLowercase]{DejaVu Sans}
\setmonofont[Scale=MatchLowercase]{DejaVu Sans Mono}


\setstretch{1.1} % line spacing

\expandafter\def\expandafter\quote\expandafter{\quote\small} % smaller quotations font

% orphan & widow control
%\clubpenalty 10000
%\widowpenalty 10000

% gaps between text and footnotes
\def\footnoteskip#1{
  \renewcommand\footnoterule{
     \vspace{#1}
     \hrule width 0.4\columnwidth%
     \vspace{3pt}
}
}
\footnoteskip{0.8em}


\setcounter{tocdepth}{2}
\setcounter{secnumdepth}{2}

%% cutting down warnings
%\hfuzz=2pt
%\hbadness=10000

% force-ordering citations according to dummy keys
\newcommand{\dummybiborderkey}[1]{}

%%% This file contains definitions of various useful macros and environments %%%
%%% Please add more macros here instead of cluttering other files with them. %%%

%%% Minor tweaks of style

% This macro defines a chapter, which is not numbered, but is included
% in the table of contents.
\def\chapwithtoc#1{
\chapter*{#1}
\addcontentsline{toc}{chapter}{#1}
}

% Draw black "slugs" whenever a line overflows, so that we can spot it easily.
\overfullrule=1mm


%%% The field of all real and natural numbers
\newcommand{\R}{\mathbb{R}}
\newcommand{\N}{\mathbb{N}}

%%% Useful operators for statistics and probability
\DeclareMathOperator{\pr}{\textsf{P}}
\DeclareMathOperator{\E}{\textsf{E}\,}
\DeclareMathOperator{\var}{\textrm{var}}
\DeclareMathOperator{\sd}{\textrm{sd}}

\DeclareMathOperator*{\argmin}{argmin}
\DeclareMathOperator*{\argmax}{argmax}
\DeclareMathOperator*{\softmax}{softmax}

\DeclareMathOperator{\relu}{ReLu}
\let\emptyset\varnothing{}

\DeclareMathOperator*{\sgn}{sgn}
\DeclareMathOperator{\attn}{Attn}
\DeclareMathOperator{\multihead}{Multihead}
\DeclareMathOperator{\concat}{concat}

\DeclarePairedDelimiter\ceil{\lceil}{\rceil}
\DeclarePairedDelimiter\floor{\lfloor}{\rfloor}

%%% Transposition of a vector/matrix
\newcommand{\T}[1]{#1^\top}

%%% Various math goodies
\newcommand{\goto}{\rightarrow}
\newcommand{\gotop}{\stackrel{P}{\longrightarrow}}
\newcommand{\maon}[1]{o(n^{#1})}
\newcommand{\abs}[1]{\left|{#1}\right|}
\newcommand{\dint}{\int_0^\tau\!\!\int_0^\tau}
\newcommand{\isqr}[1]{\frac{1}{\sqrt{#1}}}

%%% Various table goodies
\newcommand{\pulrad}[1]{\raisebox{1.5ex}[0pt]{#1}}
\newcommand{\mc}[1]{\multicolumn{1}{c}{#1}}
\newcommand{\mcl}[1]{\multicolumn{1}{l}{#1}}

% this is for the examples
\newcommand{\redund}[1]{\textcolor{black!30!red!100}{\underline{#1}}}
\newcommand{\greenund}[1]{\textcolor{black!40!green!100}{\underline{#1}}}
\newcommand{\blueund}[1]{\textcolor{black!30!blue!100}{\underline{#1}}}

% check and cross marks
\newcommand{\cmark}{\ding{51}}%
\newcommand{\xmark}{\ding{55}}%

\newcommand{\veryshortarrow}[1][3pt]{\mathrel{%
     \vcenter{\hbox{\rule[-.5\fontdimen8\textfont3]{#1}{\fontdimen8\textfont3}}}%
     \mkern-4mu\hbox{\usefont{U}{lasy}{m}{n}\symbol{41}}}}

\newcommand{\paperdisclaim}[1]{%
\begin{center}\begin{minipage}{0.9\textwidth}
\footnotesize\it #1
\end{minipage}\end{center}
}

\def\ignorecolumn#1\unskip{}

\title{Data-to-Text Generation with Neural Language Models}
% \title{Techniques for Neural Data-to-Text Generation}

\def\fulldate{}
\author{Zdeněk Kasner}
\date{2024}
\dept{Institute of Formal and Applied Linguistics}
\supervisor{Mgr. et Mgr. Ondřej Dušek, Ph.D.}
\studyprogram{Computer Science}
\studyfield{Computational Linguistics}


\begin{document}

%
%
%
\renewcommand{\thepage}{\roman{page}}
\renewcommand\cite{\citep}
\selectlanguage{english}
\maketitle

\pagestyle{plain}
\normalsize
\setcounter{page}{2}

\cleardoublepage{}
\ \vspace{10mm}

\noindent \it

\vspace{\fill}
\noindent \rm
I declare that I carried out this doctoral thesis independently,
and only with the cited sources, literature and other professional sources.

I understand that my work relates to the rights and obligations
under the Act No.~121/2000 Coll., the Copyright Act, as amended,
in particular the fact that Charles University has the right
to conclude a license agreement on the use of this work as a school work
pursuant to Section~60 paragraph~1 of the Copyright Act.

\vspace{2cm}
\noindent Prague, \today \hspace{\fill}\theauthor % doplňte patřičné
% datum, jméno a
% příjmení

%%%   Do not forget to SIGN the printed book here!
%%%                  *********


\cleardoublepage{} % new page
\pagestyle{plain}

\addcontentsline{toc}{chapter}{English Abstract}

%\selectlanguage{english}
\begin{description}[leftmargin=7.5em,labelwidth=7em,labelindent=0em,labelsep=0.5em]
  \item[Title:] \thetitle{}
  \item[Author:] \theauthor{}
  \item[Department:] \thedept{}
  \item[Supervisor:] \thesupervisor{},\\ \thedept{}
\end{description}
\subsubsection{Abstract:}

Data-to-text generation systems need to produce texts with high levels of semantic accuracy. Rule-based systems can guarantee this aspect, but their fluency and adaptability to new domains remain limited. Meanwhile, neural language models can easily generate fluent texts and adapt to new domains but are notoriously prone to producing inaccurate outputs. In this thesis, we explore how to efficiently employ neural components in data-to-text generation systems to get the best of both worlds. We focus on approaches based on pretrained transformer language models. Primarily, the models serve us as building blocks for robust and data-efficient data-to-text generation systems. Along with that, we introduce novel model-based evaluation metrics, focusing on detecting errors in data-to-text outputs, and a toolkit for preprocessing and visualizing data-to-text generation datasets. We also analyze the behavior of pretrained and large language models in specific scenarios, including describing individual relations in knowledge graphs and generating texts from standard data formats. We conclude that while employing neural language models in data-to-text generation remains a delicate endeavor, neural components can improve the fluency of the output texts and make the systems adaptable to new domains and input formats. At the same time, the semantic accuracy of the outputs can remain high if the models are used for specific, well-defined subtasks for improving text quality. For future research, we emphasize the need for proper benchmarking with suitable evaluation metrics on real-world use cases.

\begin{description}[leftmargin=7.5em,labelwidth=7em,labelindent=0em,labelsep=0.5em]
  %
  \item[Keywords:] TODO
    %
\end{description}


\cleardoublepage{}
\addcontentsline{toc}{chapter}{Czech Abstract}
\selectlanguage{czech}
\begin{description}[leftmargin=7.5em,labelwidth=7em,labelindent=0em,labelsep=0.5em]
  \item[Název práce:] TODO
  \item[Autor:] \theauthor{}
  \item[Katedra:] Ústav formální a aplikované lingvistiky
  \item[Vedoucí práce:] \thesupervisor,\\ Ústav formální a aplikované lingvistiky
\end{description}

\subsubsection{Abstrakt:}

Systémy pro generování textu z dat by měly generovat texty odpovídající co nejpřesněji vstupním datům. Pravidlové systémy tento aspekt zaručují, ale zaostávají v plynulosti výstupů a možnostech přizpůsobení pro nové domény. Naopak neuronové jazykové modely zvládají snadno generovat plynulé texty a přizpůsobovat se novým doménám, ale jsou notoricky náchylné k produkci nepřesných výstupů. V této práci zkoumáme, jak efektivně zakomponovat do systémů pro generování textu z dat neuronové modely tak, abychom propojili výhody obou typů systémů. Naše přístupy zakládáme na předtrénovaných jazykových modelech architektury transformer. Tyto modely primárně používáme jako stavební bloky, díky kterým mohou být systémy pro generování textu robustní a efektivně se učit z trénovacích dat. Spolu s tím představujeme nové evaluační metriky pro odhalování chyb ve výstupech, založené na předtrénovaných modelech, a sadu nástrojů pro předzpracování a vizualizaci datasetů pro generování textu z dat. Analyzujeme také chování předtrénovaných a velkých jazykových modelů ve specifických případech jako je popis jednotlivých relaci ve znalostních grafech a generování textů ze standardních datových formátů. Z~našich experimentů vyplývá, že ačkoli k použití neuronových jazykových modelů při generování textu z dat je potřeba přistupovat s rozmyslem, neuronové komponenty mohou zlepšit plynulost výstupních textů a umožnit přizpůsobení systémů novým doménám. Přesnost výstupů přitom může zůstat vysoká, pokud se modely používají pro konkrétní, přesně definované dílčí úkoly pro zlepšení kvality textu. Do budoucna zdůrazňujeme nutnost vyhodnocovat systémy pomocí vhodných evaluačních metrik na reálných problémech.

\begin{description}[leftmargin=7.5em,labelwidth=7em,labelindent=0em,labelsep=0.5em]
  %
  \item[Klíčová slova:] TODO
    %
\end{description}

\selectlanguage{english}




\cleardoublepage{}
\ \vspace{10mm}

\addcontentsline{toc}{chapter}{Acknowledgements}
\subsection*{Acknowledgements}

{

  TODO
  % Here, you can thank anyone and say anything.

  %   \vspace{1\baselineskip}
  %   \noindent
  %   This is how I separated different kinds of thank-yous.

  %   \vspace{1\baselineskip}
  %   \noindent
  %   ... continued. 
}

\vfill


{\noindent\footnotesize %
  This work has been using language resources and tools developed and/or stored and/or distributed by the  LINDAT/CLARIN project of the Ministry of Education, Youth and Sports of the Czech Republic (project LM2015071).
}

\cleardoublepage{}
\addcontentsline{toc}{chapter}{Table of Contents}
\tableofcontents % automatically generated

\cleardoublepage{}
\renewcommand{\chapterheadstartvskip}{\vspace*{-10mm}} % chapter spacing
\setstretch{1.2} % line spacing

%
% TEXT START
%
\renewcommand{\thepage}{\arabic{page}}
\setcounter{page}{1}




\sloppy
% % %%%%%%%%%%%%%%%%%%%%%%%%%%%%%%%%%%%%%%%%%%%%%%%%%%%%%%%%%%%%%%%%%%%%%%%%%%%%
\chapter{Introduction}
\label{chap:intro}
Producing \emph{natural language} comes \emph{natural} primarily to us, humans.
The key to computers' versatility and efficiency---their ``language''---are data structures: arrays, lists, trees and graphs, tables and databases.
We can scrutinize the data with appropriate tools---provided sufficient domain expertise and enough time---but this does not address the core of the problem: that to most people, reading structured data is like trying to decipher a foreign language. As the volume of data in our world grows, it is tempting to turn the question on its head: Can we instead teach the computers to describe the structured data in our, natural language?


This question has been addressed since the dawn of computing. The first attempts at producing natural language with a computer date back to the audacious attempts of \emph{translating} between English and Russian in 1950's \cite{sheridan1955research} which stirred a lot of excitement, and lead to a belief that \emph{generating} English sentences with a set of rules is a simpler task. Although in 1960's, people slowly began to ponder on its difficulties---\citet{yngve1961random} notes even the first ten sentences of a children's book provide \emph{``surprisingly wide linguistic diversity''} for assembling appropriate grammar rules---the overall sentiment was that language generation will soon be solved. The seminal work of \citet{winograd1971procedures}, describing in 461 pages the SHRDLU system which manipulates blocks in an imaginary block world according to user instructions, only glosses over presenting the state of the world to the user:
\begin{pquotation}{\citealp[p.384]{winograd1971procedures}}
    [R]esponses can be made as complex and varied as we want, since they are created by the programmer, and the program only repeats them.
\end{pquotation}
In other words: If we can make the computers mechanically repeat whatever we say, what else is there to generating language?

Fast forward to the present, the research world is beaming with excitement again: neural \acp{lm} have a suprising ability of producing the long-sought \emph{complex} and \emph{varied} language \cite{radford2019language,brown2020language}. Similarly to other tasks in \ac{ai}---from object recognition \cite{papert1966summer} to self-driving cars \cite{autonomouscars}---the apparent ease of the task for humans has proven deceptive. In the end, ot took us 50 years to build tools for generating fluent language. To make progress, we had to shift our attention from linguistic theories and rule-based systems, re-defining our systems in terms of data-based approaches and generic learning algorithms.

In the previous decades, in which language was generated using rules and grammars, the goal of systems for automatic \ac{nlg}---which has meanwhile established itself as a standalone scientific discipline, with its journals, conferences, and stable base of researchers \cite{ACLanthologySIGGEN}---was rather pragmatic.
% taking structured data from a particular system  and presenting it to the users in the form they will understand. 
The works in \ac{nlg} were the works of \emph{engineering}: natural language was simply taken as one of the suitable mediums to present the structured data to the users in an understandable form. From chart captioning systems \cite{mittalDescribingComplexCharts1998} and graph descriptors \cite{sunDomainIndependentSentence2006}, to weather forecast systems \cite{belzAutomaticGenerationWeather2008} and healthcare report generators \cite{portetAutomaticGenerationTextual2009}, the research papers read like \emph{how-to's} for building robust systems with widely adopted tools. As a result, the \ac{nlg} systems from that time were accurate and reliable, if only a bit too domain-specific and rigid \cite{reiterBuildingAppliedNatural1997,gattSurveyStateArt2018}.

% The essence of these systems was transforming a non-linguistic inputs to linguistic outputs according to a sequence of rule. In some sense, we can therefore say that transforming data to text---what is now specifically called \ac{d2t} generation---was all there was to \ac{nlg}.

With neural models, \ac{nlp} as a research field (along with \ac{nlg} as one of its subfields) has changed. Most notably, it has become more experimental. Although neural \acp{lm} opened up fascinating possibilities in building end-to-end systems and solving the long-standing issues with fluency and domain-independence \cite{ferreiraNeuralDatatotextGeneration2019,dusekEvaluatingStateoftheartEndtoEnd2020,sharmaInnovationsNeuralDatatotext2022}, working with neural models turned out to be closer to behavioral sciences than engineering \cite{holtzmanGenerativeModelsComplex2023}. As the researchers began adapting to the change in the paradigm, the issues with respect to experimental design and evaluation came to surface \cite{gehrmannRepairingCrackedFoundation2022} and the whole change may have been percieved as a step back \cite{reiter2020academic}. The shift towards experimental approaches has also created a gap between research and industry; the industry opting for established approaches meeting industrial standards
% in the ever-changing research landscape
\cite{daleNaturalLanguageGeneration2020,daleNavigatingTextGeneration2023}.


Nevertheless, the progressive approach adopted by \ac{nlp} over the past few years turned out to have its merits. The general emphasis on open research, inherited from the \ac{ml} field---where publicly releasing papers, code, and models has become commonplace---has allowed everybody to stand on the proverbial shoulders of giants. As people can build on others' code within minutes since its publication, the research is accelerating and gathering more observations. The convergence towards generic aproaches has also lead to heavy cross-pollination of ideas, i.e., making ideas for specific tasks applicable to other tasks. As such, \ac{nlg} is helping to advance other areas of \ac{nlp} and contribute to general knowledge on natural language, its production and processing.

Finally, as we gained other ways to generate language than from structured data---summarize and paraphrase texts, continue text segments, generate stories and answers to questions, or describe images and videos---% \cite{Dong2021ASO}
the original field dealing with generating descriptions of structured data has gradually adopted the---perhaps more apt---name of \emph{\ac{d2t} generation}.

This thesis is a story about how \acl{d2t} generation and neural \aclp{lm} came together, or at least are seeking common ground. On the way, it touches various facets of \ac{d2t} generation, from improving the generation in low-resource settings (\autoref{chap:low-res}), evaluating generated texts (\autoref{chap:evaluation}), processing and visualizing data (\autoref{chap:tabgenie}), to interpreting system behavior (\autoref{chap:investigating}). It tries to make the point that the adoption of neural approaches in \ac{d2t}---although it has not been exactly smooth (and it would be a stretch to call it a \emph{symbiosis})---is driven by the promise that it can help us to solve some long-stading issues. The thesis also inevitably reflects the shifts in \ac{nlp} between 2020 and 2024: from early (sometimes a bit desperate) attempts at generating fluent language with small pretrained \acp{lm}, all the way up to dealing with the hype surrounding the \acp{llm}.  Far from being a complete survey, the thesis is rather explorative, but can hopefully offer both pointers to newcomers in the field along with a handful of unorthodox ideas.










\section{Motivation}
\label{sec:rq}

The main goal of the thesis is to close the gap outlined in the introduction: turning experimental approaches into reliable and accurate \ac{d2t} generation systems. For the premise, we take neural \acp{lm}\footnote{For brevity, we will use ``\acp{lm}'' to denote ``neural \acp{lm}'' throughout the work, unless stated otherwise.} as a mean of producing fluent and natural-sounding text. However, instead of considering \acp{lm} as a hammer and everything as a nail, we carefully study how to integrate \acp{lm} in \ac{d2t} systems while adhering to all the strict demands on fluency, controllability, and semantic accuracy of the output.

The auxiliary goal of the thesis is then to \textit{understand}: understand the data we are dealing with, the outputs we can reasonably expect, and the behavior of neural-based systems in certain conditions. The approaches presented for \ac{d2t} generation can often be generalized to other subfield, even though \ac{d2t} generation has several specifics which makes it a good subject for this kind of study: its low-resourceness (due to which---thankfully---there are questions that cannot be answered with ``\emph{scaling}''), the tension between established rule-based approaches and the new-coming neural approaches, and, admittedly, the fact that it is the field that we have gathered the most expertise in.

To make these goals more tangible, the thesis addresses the following research questions:

\begin{description}
    \item[RQ1] \textbf{In which scenarios are \acp{lm} useful for \ac{d2t} generation?} For starters, it is crucial to identify the strong sides of \acp{lm} and get an intuition of where the models can make the most impact. How far can we get with \ac{lm}-only baselines? Are there outcome that we can get with \ac{lm} which are better than any previous approaches? At the same time, it is essential to understand their limitations, and be able to employ the \acp{lm} only where it is actually needed.
    \item[RQ2] \textbf{How to efficiently process the structured data with \acp{lm}?} With any kind of structured data, we need to deal with the fact that \acp{lm} were pre-trained on modeling plain text, while the data have inner structure. In order to efficiently leverage the knowledge in \acp{lm}---especially in low-resource settings---we need to find the way to transform the data into a suitable input format while keeping the structure of the data (along with other information) intact.
    \item[RQ3] \textbf{How to make \acp{lm}-based systems more controllable?} Any neural component introduced in the \ac{d2t} generation system will raise issues with controllability. The question is if we can minimize these issues, for example by building systems out of smaller and simpler components, training the models for more predictable tasks, or producing intermediate outputs which can be manually examined.
    \item[RQ4] \textbf{How to evaluate the outputs of \ac{d2t} generation systems?} Evaluating generated text is hard, and it gets even harder as the quality of the texts starts to approach human level. Since human evaluation is costly and time-consuming, we study how to build automatic metrics for researchers and the system developer, focusing on the most pressing issue in \ac{d2t} generation: the semantic accuracy of the generated texts.
    \item[RQ5] \textbf{How do the neural \ac{d2t} generation systems behave?} Undestanding the abilities and limitations of the systems we are building is crucial both for researchers and for potentional practicioners in the field. We are interested in measuring the abilities of \ac{lm}-based systems and pointing out their weak spots.
\end{description}



\section{Main Contributions}
\label{sec:contributions}


Our main contributions, corresponding numerically to the research questions outlined above, are:
\begin{enumerate}
    \item We show that with a very \textbf{simple \ac{lm}-based finetuned baseline}, we can achieve strong results on a shared task of generating texts from a knowledge graph (\autoref{sec:finetuning}). We also point out the advantages and limitations of open \acp{llm} on \ac{d2t} generation in zero-shot settings (\autoref{sec:prompting}).
    \item We show how to \textbf{transform the data to intermediate text-like input} suitable for \acp{lm} using hand-crafted or automatically extracted templates (\Cref{sec:iterative,sec:pipeline,sec:sem-acc}), rule-based \ac{nlg} methods (\autoref{sec:eval-token}), and specialized \acp{lm} (\autoref{sec:describing}). We show that these methods can serve as a basis both for competitive neural-based \ac{d2t} generation systems and for novel \ac{lm}-based evaluation metrics.
    \item We show how we can limit \acp{lm} to the task of \textit{improving text fluency}, and use these \acp{lm} for building \textbf{more controllable \ac{d2t} generation systems} with an iterative approach (\autoref{sec:iterative}) and modular architecture (\autoref{sec:pipeline}).
    \item We develop \textbf{\ac{lm}-based automatic metrics} for evaluating outputs of \ac{d2t} generation systems on the level of data item mentions (\autoref{sec:sem-acc}) and output tokens (\autoref{sec:eval-token}), showing strong results regarding correlation with human judgement in comparison with other available metrics.
    \item We build a \textbf{tool for visualizing} the structured data and model outputs and show how we can unify the format of mutliple \ac{d2t} generation dataset for easier processing (\autoref{sec:tabgenie}). We also \textbf{study the behaviors of open \acp{llm}} across multiple \ac{d2t} tasks, data formats, and domains, evaluate their semantic accuracy, and provide recommendations for future research (\Cref{sec:prompting,sec:describing}).
\end{enumerate}



\section{Thesis Overview}
\label{sec:overview}
% \subsection*{\autoref{chap:low-res}}

% \begin{itemize}
%     \item Finetuning LMs: Train Hard, Finetune Easy: Multilingual Denoising for RDF-to-Text Generation \cite{kasnerTrainHardFinetune2020}
% \end{itemize}

\begin{table*}[ht]
    \small
    \begin{tabular}{p{4cm}p{7.5cm}p{1cm}}
        \toprule
        \textbf{Publication}                             & \textbf{Author contribution} & \textbf{Sec.}         \\ \midrule
        \multicolumn{3}{l}{\textbf{\autoref{chap:low-res}: Low-Resource Data-to-Text Generation}}               \\
        \citet{kasnerTrainHardFinetune2020}              & TODO                         & §\ref{sec:finetuning} \\
        \citet{kasnerDatatoTextGenerationIterative2020}  & TODO                         & §\ref{sec:iterative}  \\
        \citet{kasner2022neural}                         & TODO                         & §\ref{sec:pipeline}   \\ \cdashlinelr{1-3}
        \multicolumn{3}{l}{\textbf{\autoref{chap:evaluation}: Evaluating Generated Text}}                       \\
        \citet{dusekEvaluatingSemanticAccuracy2020}      & TODO                         & §\ref{sec:sem-acc}    \\
        \citet{kasnerTextinContextTokenLevelError2021}   & TODO                         & §\ref{sec:eval-token} \\ \cdashlinelr{1-3}
        \multicolumn{3}{l}{\textbf{\autoref{chap:tabgenie}: Data Processing and Visualization}}                 \\
        \citet{kasnerTabGenieToolkitTabletoText2023}     & TODO                         & §\ref{sec:tabgenie}   \\ \cdashlinelr{1-3}
        \multicolumn{3}{l}{\textbf{\autoref{chap:investigating}: Investigating Model Capabilities}}             \\
        \citet{kasnerMindLabelsDescribing2022}           & TODO                         & §\ref{sec:describing} \\
        \citet{kasnerReferenceBasedMetricsAnalyzing2024} & TODO                         & §\ref{sec:prompting}  \\\bottomrule
    \end{tabular}
\end{table*}
% % %%%%%%%%%%%%%%%%%%%%%%%%%%%%%%%%%%%%%%%%%%%%%%%%%%%%%%%%%%%%%%%%%%%%%%%%%%%%
\chapter{Method}%
\label{chap:method}
% %%%%%%%%%%%%%%%%%%%%%%%%%%%%%%%%%%%%%%%%%%%%%%%%%%%%%%%%%%%%%%%%%%%%%%%%%%%%


% % %%%%%%%%%%%%%%%%%%%%%%%%%%%%%%%%%%%%%%%%%%%%%%%%%%%%%%%%%%%%%%%%%%%%%%%%%%%%
\chapter{Conclusions}%
\label{chap:conslusions}
% %%%%%%%%%%%%%%%%%%%%%%%%%%%%%%%%%%%%%%%%%%%%%%%%%%%%%%%%%%%%%%%%%%%%%%%%%%%%

\chapter{Introduction}
The key to computers' versatility and efficiency -- their ``language'' -- is processing, producing, and storing structured data: arrays, lists, tree- and graph-like structures, tables and databases. Producing \textit{natural language} is in fact \textit{natural} only to humans, and to the computers less so. Once the output is computed and eventually meets human eyes,  it needs to be understood and interpreted. Should we thus learn to understand the structured data, or should we teach the computers to translate the data into our language?

The answer seems to be in favor of the latter.
% From the early days of computing, there were attempts to produce output in natural language. 
In fact, after the first audacious attempts of computers \textit{translating} between English and Russian from 1950's \cite{sheridan1955research}, \textit{generating} English-only sentences based on the computer state seemed like an arguably simpler task. To take one example, the seminal work from \citet{winograd1971procedures} on SHRDLU, a system which can manipulate blocks according to natural language instructions, spending over 300 pages on achieving natural language understanding, only glosses over language generation, for which the reasoning goes like following:
\begin{quote}
  \textit{[R]esponses can be made as complex and varied as we want, since they are created by the programmer, and the program only repeats them.} \cite[p.384]{winograd1971procedures}
\end{quote}

Skipping more than 50 years ahead, the neural \glsxtrfullpl{lm} based on the Transformer architecture \cite{vaswani2017attention} are only now starting to provide the long-sought solutions for making the computer generating \textit{complex} and \textit{varied} language. Behind us are decades of works on trying to build fluent, accurate and reliable \glsxtrfull{nlg} systems \cite{reiterBuildingAppliedNatural1997,gattSurveyStateArt2018}. Similarly to other ideas in artificial intelligence -- from object recognition \cite{papert1966summer} to self-driving cars \cite{autonomouscars} -- the apparent ease of the task for humans has proven deceptive.



% Automating natural language generation is therefore one of the holy grails of \acl{ai}.


% Data-to-text generation is the process of starting with the textual representation understandable to computers and ending up the the  textual representation undestandable to humans.

\section{Research Questions}
\section{Main Contributions}
\section{Thesis Overview}

\chapter{Background}
\section{Neural Language Models}
\subsection{Neural Networks}
\subsection{Transformer Architecture}
\section{Pretrained Language Models}
\subsection{Large Language Models}
\section{Data-to-Text Generation}
\subsection{Pipeline-based Approaches}
\subsection{Neural Approaches}
\subsection{Datasets}
\section{NLG Evaluation}
\subsection{Classical Metrics}
\subsection{Model-based Metrics}

\chapter{Low-Resource Data-to-Text Generation}
\section{Motivation}
\section{Finetuning LMs}
\subsection{WebNLG+ Shared Task}
\subsection{Our Submission}
\section{Iterative Template Fusion with Text-Editing LMs}
\subsection{Text-Editing LMs}
\subsection{Experiments}
\section{Pipelined Text-Based Operations with Pretrained LMs}
\subsection{Pipeline Operations}
\subsection{Experiments}
% \section{Zero-shot Prompting with LLMs}

\chapter{Evaluating Generated Text}
\section{Motivation}
\section{Evaluating Semantic Accuracy}
\subsection{Experiments}
\section{Token-Level Error Detection}
\subsection{Shared Task}
\subsection{Our Submission}

\chapter{Data Processing and Visualization}
\section{Motivation}
\section{TabGenie Toolkit}
\subsection{Data Processing}
\subsection{Web Interface}
\subsection{Programming Interface}


\chapter{Investigating Model Capabilities}
\section{Motivation}
\section{Describing Triples in Knowledge Graphs}
\subsection{Knowledge Graphs}
\subsection{Rel2Text Dataset}
\subsection{Experiments}
\section{Prompting Open LLMs}
\subsection{\textsc{Quintd} Toolkit}
\subsection{Experiments}




\chapter{Conclusions}


%
% TEXT END
%

\renewcommand{\chapterheadstartvskip}{\vspace*{0mm}} % chapter spacing

\cleardoublepage{}
\bibliographystyle{csplainnat}
\addcontentsline{toc}{chapter}{Bibliography}
{\small \bibliography{references}}

\cleardoublepage{}
\addcontentsline{toc}{chapter}{List of Abbreviations}
\renewcommand*{\acronymname}{List of Abbreviations}
\printglossary[type=\acronymtype,style=index]

\addcontentsline{toc}{chapter}{List of Tables}
{\small \listoftables\par}

\addcontentsline{toc}{chapter}{List of Figures}
{\small \listoffigures\par}

\cleardoublepage{}
\addcontentsline{toc}{chapter}{List of Publications}
\chapter*{List of Publications}

\phantom{\nobibliography*{references}}

% -----------------------------------------------------------------------------

\noindent\bibentry{kasnerTrainHardFinetune2020}
\begin{itemize}[noitemsep,topsep=0pt]

    \item The data-to-text generation system based on the finetuned mBART model (\autoref{sec:finetuning}).
    \item Our submission for the WebNLG+ shared task.
    \item Citations (without self-citations): 9
\end{itemize}\vspace{.5\baselineskip}

% -----------------------------------------------------------------------------
\noindent\bibentry{kasnerDatatoTextGenerationIterative2020}
\begin{itemize}[noitemsep,topsep=0pt]
    \item The data-to-text generation system based on iterative text editing (\autoref{sec:iterative}).
    \item Citations (without self-citations): 17

\end{itemize}\vspace{.5\baselineskip}

% -----------------------------------------------------------------------------
\noindent\bibentry{kasner2022neural}
\begin{itemize}[noitemsep,topsep=0pt]
    \item The data-to-text generation system based on a pipeline of neural modules (\autoref{sec:pipeline}).
    \item Citations (without self-citations): 21

\end{itemize}\vspace{.5\baselineskip}

% -----------------------------------------------------------------------------
\noindent\bibentry{dusekEvaluatingSemanticAccuracy2020}
\begin{itemize}[noitemsep,topsep=0pt]

    \item The metric for detecting omissions and hallucinations in generated texts (\autoref{sec:tok-eval}).
    \item Best short paper at INLG 2020.
    \item Citations (without self-citations): 47

\end{itemize}\vspace{.5\baselineskip}


% -----------------------------------------------------------------------------
\noindent\bibentry{kasnerTextinContextTokenLevelError2021}
\begin{itemize}[noitemsep,topsep=0pt]
    \item The metric for token-level error detection in generated texts (\autoref{sec:tok-eval})
    \item Our submission to the shared task Evaluating Accuracy in Generated Texts.
    \item Citations (without self-citations): 6

\end{itemize}\vspace{.5\baselineskip}


% -----------------------------------------------------------------------------
\noindent\bibentry{kasnerTabGenieToolkitTabletoText2023}
\begin{itemize}[noitemsep,topsep=0pt]

    \item The toolkit for processing and visualization of data-to-text generation datasets (\autoref{sec:tabgenie}).
    \item Citations (without self-citations): 2

\end{itemize}\vspace{.5\baselineskip}


% -----------------------------------------------------------------------------
\noindent\bibentry{kasnerMindLabelsDescribing2022}
\begin{itemize}[noitemsep,topsep=0pt]

    \item The analysis of verbalizing relations in knowledge graphs with pretrained language models (\autoref{sec:rel2text}).
    \item Citations (without self-citations): 3

\end{itemize}\vspace{.5\baselineskip}


% -----------------------------------------------------------------------------
\noindent\bibentry{kasnerReferenceBasedMetricsAnalyzing2024}
\begin{itemize}[noitemsep,topsep=0pt]

    \item The analysis of data-to-text generation with open large language models (\autoref{sec:quintd}).
    \item Citations (without self-citations): 2

\end{itemize}\vspace{.5\baselineskip}

\vfill

\noindent Only publications relevant to this thesis are included. The number of
citations was computed using Semantic Scholar API. Total number of citations of
publications related to the topic of the thesis (without self-citations) by June 14, 2024: {\bf 107}.


\end{document}
