
\chapter{Background}
\label{chap:background}

This chapter explains the \textbf{basic concepts} used throughout the thesis. First, we explain \textbf{neural \acp{lm}}: from the basic of neural networks (§\ref{sec:nns}) and language modeling (§\ref{sec:lm-basics}), up to pretrained (§\ref{sec:plms}) and large language models (§\ref{sec:llms}). Next, we move on to \textbf{\ac{d2t} generation}: covering rule-based (§\ref{sec:rule-d2t}) and neural-based systems (§\ref{sec:neural-d2t}), \ac{d2t} datasets (§\ref{sec:datasets}) and evaluation methods (§\ref{sec:evaluation}). We assume that the reader has certain expertise in related areas of \ac{nlp}, although not necessarily in \ac{nlg}. We also aim to make the work self-contained by covering all the related concepts, but the interested reader is referred to the respective works for details.

Besides explaining the basic concepts, the chapter serves also as an \textbf{overview of the state of the art} in the field. In particular, the later subsections (\Cref{sec:plms,sec:llms} for neural \acp{lm} and \Cref{sec:neural-d2t,sec:datasets,sec:evaluation} for \ac{d2t} generation) focus on providing pointers to related work and describing the datasets and models used for the experiments. As such, the chapter serves as a main reference for the related work, and we will only briefly revisit the most relevant works in the respective chapters.


\section{Neural Language Models}
\label{sec:lms}
In this section, we work our way towards neural \acp{lm}: the mathematical foundations of \acp{nn} on which the neural \acp{lm} are built on, and the way \acp{lm} are constructed, trained, and eventually applied in \ac{nlp}.

\subsection{Neural Networks}
\label{sec:nns}
First, we need to build a tool for learning patterns from data\footnote{Until we get to \ac{d2t} generation in \autoref{sec:d2t}, we will use the word ``\textit{data}'' only in its abstract sense, as in ``any inputs we can apply our algorithms to''. We will use the term ``structured data'' whenever it is necessary to make the distinction.}. This tool---which for us will be the \textbf{neural networks}---will later help us learning patterns about language from large-scale data, and eventually generating the language.

Let's say our goal is to predict the real-number output $y \in \mathbb{R}$ for the given real vector $\mathbf{x} \in \mathbb{R}$.\footnote{We will follow the convention that vectors are denoted with boldface letters ($\mathbf{x}$), and real numbers with plain letters ($x$).} Let's also assume that the $\mathbf{x} \rightarrow y$ mapping is not arbitrary (that would leave us with memorizing all the $(\mathbf{x},y)$ pairs), but follows some regularities and underlying patterns that can be learned. Since we usually consider $(\mathbf{x},y)$ to be representations of real-word data, e.g. documents and their labels, this assumption will be naturally satisfied.

For learning the underlying connection between $\mathbf{x}$ and $y$, we will use the mathematical models designed to capture the connection in their parameters. The idea is that the models estimate the parameters from a limited set of examples called the \textit{training data}:  $\mathcal{D_{\text{train}}} = \{(\mathbf{x}_1, y_1), \ldots, (\mathbf{x}_{n}, y_{n})\}$, and use the learned parameters to predict the outputs on the \textit{test data}: $\mathcal{D_{\text{test}}} = \{(\mathbf{x}_1, y_1), \ldots, (\mathbf{x}_{m}, y_{m})\}$.

\paragraph{Perceptron Algorithm} One of the early mathematical models designed for predicting the outputs based on the inputs is the \emph{perceptron algorithm} \cite{rosenblatt1958perceptron}. In this case, we assume the output is binary: $y \in \{-1, 1\}$. The algorithm learns parameters $\textbf{w}$ and $b$ describing a linear decision boundary, separating the data points $\mathbf{x_i}$ so that the points $\{\mathbf{x_i} | y_i = -1\}$ lay on one side of the boundary and the points $\{\mathbf{x_j} | y_j = 1\}$ on another. The algorithm proceeds as follows:


\begin{enumerate}
    \item The parameters $\textbf{w}$ and $b$ are initialized to small random values or zeros.
    \item For each training example $(\mathbf{x}_i, y_i)$:
          \begin{itemize}
              \item The algorithm computes the predicted output $\hat{y}_i$ using the current weights and bias: $\hat{y}_i = \text{sign}(\textbf{w} \cdot x_i + b)$.
              \item The algorithm updates the weights and bias:
                    \[ \textbf{w} = \textbf{w} + (y_i - \hat{y}_i) \textbf{x}_i \]
                    \[ b = b + y_i - \hat{y}_i \]
          \end{itemize}
    \item This process is repeated until convergence.
\end{enumerate}

The perceptron algorithm is guaranteed to converge if the data is linearly separable, i.e., if there exists a hyperplane which separates the data belonging to one class from another \cite{novikoff1962convergence}.

\paragraph{Multi-layer Perceptron} To overcome the limitations of the perceptron algoritm, we can build a network of interconnected units

\subsection{Language Models}
\label{sec:lm-basics}
\subsection{Transformer Architecture}
\label{sec:transformer}
\subsection{Pretrained Language Models}
\label{sec:plms}
\subsection{Large Language Models}
\label{sec:llms}
\section{Data-to-Text Generation}
\label{sec:d2t}
\subsection{Rule-based Approaches}
\label{sec:rule-d2t}
\subsection{Neural Approaches}
\label{sec:neural-d2t}
\subsection{Datasets}
\label{sec:datasets}
\subsection{Evaluation Metrics}
\label{sec:evaluation}