
\chapter{Background}
\label{chap:background}

This chapter explains the \textbf{basic concepts} used throughout the thesis. First, we explain \textbf{neural \acp{lm}}: from the basic of neural networks (§\ref{sec:nns}) and language modeling (§\ref{sec:lm-basics}), up to pretrained (§\ref{sec:plms}) and large language models (§\ref{sec:llms}). Next, we move on to \textbf{\ac{d2t} generation}: covering rule-based (§\ref{sec:rule-d2t}) and neural-based systems (§\ref{sec:neural-d2t}), \ac{d2t} datasets (§\ref{sec:datasets}) and evaluation methods (§\ref{sec:evaluation}). We assume that the reader has certain expertise in related areas of \ac{nlp}, although not necessarily in \ac{nlg}. We also aim to make the work self-contained by covering all the related concepts, but the explanations are brief, and the interested reader is referred to the respective works for details.

Besides explaining the basic concepts, the chapter serves also as an \textbf{overview of the state of the art} in the field. In particular, the later subsections (\Cref{sec:plms,sec:llms} for neural \acp{lm} and \Cref{sec:neural-d2t,sec:datasets,sec:evaluation} for \ac{d2t} generation) focus on summarizing related work, providing pointers to it, and describing the datasets and models used for the experiments. As such, the chapter serves as a main reference for the related work for later chapters, and we will only briefly revisit the most relevant works in the respective chapters.


\section{Neural Language Models}
\label{sec:lms}
In this section, we work our way towards neural \acp{lm}: the mathematical foundations of \acp{nn} the \acp{lm} are built on, and the way \acp{lm} are constructed, trained, and eventually applied in \ac{nlp}.

\subsection{Neural Networks}
\label{sec:nns}
Let's assume our goal is to predict the real-number output $y \in \mathbb{R}$ for the given real vector $\mathbf{x} \in \mathbb{R}$. For an arbitrary $\mathbf{x} \rightarrow y$ mapping, that is quite an ambitious goal. To make the task easier, let's also assume that $\mathbf{x}$'s and $y$'s are representations of some real-world data and there is a certain non-trivial dependence of values of $y$ based on $\mathbf{x}$ which can be learned.
%  and come from naturally occuring distributions $\mathbf{x} \in \mathcal{X}$, $y \in \mathcal{Y}$. 
% This suggest there are regularities and underlying patterns in the data and .

For learning how $y$ depends on $\mathbf{x}$ (which will subsequentially allow us to make predictions about the real world phenomena represented by $\mathbf{x}$'s and $y$'s), we can use mathematical models designed to capture these regularities and underlying patterns in their parameters. The idea is that the models learn the patterns from a limited set of examples called the \textit{training data},  $\mathcal{D_{\text{train}}} = \{(\mathbf{x}_1, y_1), \ldots, (\mathbf{x}_{n}, y_{n})\}$, and use the learned parameters to predict the outputs on the \textit{test data}, $\mathcal{D_{\text{test}}} = \{(\mathbf{x}_1, y_1), \ldots, (\mathbf{x}_{m}, y_{m})\}$.

\paragraph{Perceptron Algorithm} One of the early mathematical models designed for predicting these outputs (which are in this case binary, $y \in \{-1, 1\}$) from the data is the perceptron algorithm. The perceptron algorithm learns a linear decision boundary for separating data points with $(\mathbf{x_i}, y_i = -1)$ and $y_i = 1$. To do that, the perceptron algorithm learns a set of weights $w$ and a bias $b$ to make predictions based on the input features.




\begin{enumerate}
    \item The weights $w$ and the bias $b$ are initialized to small random values or zeros.
    \item For each training example $(\mathbf{x}_i, y_i)$:
          \begin{itemize}
              \item Compute the predicted output $\hat{y}_i$ using the current weights and bias: $\hat{y}_i = \text{sign}(w \cdot x_i + b)$.
              \item Update the weights and bias using the perceptron update rule:
                    \[ w = w + \alpha (y_i - \hat{y}_i) x_i \]
                    \[ b = b + \alpha (y_i - \hat{y}_i) \]
                    where $\alpha$ is the learning rate.
          \end{itemize}
    \item \textbf{Termination:} Repeat the iterative update until convergence or for a fixed number of iterations.
\end{enumerate}

The perceptron algorithm converges if the data is linearly separable; otherwise, it may not converge. Additionally, the learning rate $\alpha$ is a hyperparameter that influences the convergence speed and stability of the algorithm.




\subsection{Language Models}
\label{sec:lm-basics}
\subsection{Transformer Architecture}
\label{sec:transformer}
\subsection{Pretrained Language Models}
\label{sec:plms}
\subsection{Large Language Models}
\label{sec:llms}
\section{Data-to-Text Generation}
\label{sec:d2t}
\subsection{Rule-based Approaches}
\label{sec:rule-d2t}
\subsection{Neural Approaches}
\label{sec:neural-d2t}
\subsection{Datasets}
\label{sec:datasets}
\subsection{Evaluation Metrics}
\label{sec:evaluation}