
\chapter{Background}
\label{chap:background}

This chapter explains the \textbf{basic concepts} used throughout the thesis. First, we explain \textbf{neural \acp{lm}}: from the basic of neural networks (§\ref{sec:nns}) and language modeling (§\ref{sec:lm-basics}), up to pretrained (§\ref{sec:plms}) and large language models (§\ref{sec:llms}). Next, we move on to \textbf{\ac{d2t} generation}: covering rule-based (§\ref{sec:rule-d2t}) and neural-based systems (§\ref{sec:neural-d2t}), \ac{d2t} datasets (§\ref{sec:datasets}) and evaluation methods (§\ref{sec:evaluation}). We assume that the reader has certain expertise in related areas of \ac{nlp}, although not necessarily in \ac{nlg}. We also aim to make the work self-contained by covering all the related concepts, but the explanations are brief, and the interested reader is referred to the respective works for details.

Besides explaining the basic concepts, the chapter serves also as an \textbf{overview of the state of the art} in the field. In particular, the later subsections (\Cref{sec:plms,sec:llms} for neural \acp{lm} and \Cref{sec:neural-d2t,sec:datasets,sec:evaluation} for \ac{d2t} generation) focus on summarizing related work, providing pointers to it, and describing the datasets and models used for the experiments. As such, the chapter serves as a main reference for the related work for later chapters, and we will only briefly revisit the most relevant works in the respective chapters.


\section{Neural Language Models}
\label{sec:lms}
In this section, we work our way towards neural \acp{lm}: the mathematical foundations of \acp{nn} the \acp{lm} are built on, and the way \acp{lm} are constructed, trained, and eventually applied in \ac{nlp}.

\subsection{Neural Networks}
\label{sec:nns}




\subsection{Language Models}
\label{sec:lm-basics}
\subsection{Transformer Architecture}
\label{sec:transformer}
\subsection{Pretrained Language Models}
\label{sec:plms}
\subsection{Large Language Models}
\label{sec:llms}
\section{Data-to-Text Generation}
\label{sec:d2t}
\subsection{Rule-based Approaches}
\label{sec:rule-d2t}
\subsection{Neural Approaches}
\label{sec:neural-d2t}
\subsection{Datasets}
\label{sec:datasets}
\subsection{Evaluation Metrics}
\label{sec:evaluation}