
\chapter{Background}
\label{chap:background}

This chapter explains the \textbf{basic concepts} used throughout the thesis. First, we explain \textbf{neural \acp{lm}}: from the basics of neural networks (§\ref{sec:nns}) and language modeling (§\ref{sec:lm-basics}), up to pretrained (§\ref{sec:plms}) and large language models (§\ref{sec:llms}). Then we move on to \textbf{\ac{d2t} generation}: covering rule-based (§\ref{sec:rule-d2t}) and neural-based systems (§\ref{sec:neural-d2t}), \ac{d2t} datasets (§\ref{sec:datasets}) and evaluation methods (§\ref{sec:evaluation}). We assume that the reader has certain expertise in related areas of \ac{nlp}, although not necessarily in \ac{nlg}. We aim to make the work self-contained by covering all the important concepts, pointing the interested reader to related work for more details.

Besides explaining the basic concepts, the chapter serves also as an \textbf{overview of the state of the art} in the field of interest. In particular, the later subsections (\Cref{sec:plms,sec:llms} for neural \acp{lm} and \Cref{sec:neural-d2t,sec:datasets,sec:evaluation} for \ac{d2t} generation) summarize related work and describe the datasets and models used for the experiments. As such, the chapter serves as the main point of reference; we will only briefly revisit the most relevant works in the respective chapters.


\section{Neural Language Models}
\label{sec:lms}
In this section, we work our way towards neural \acp{lm}: the mathematical foundations of \acp{nn} on which the neural \acp{lm} are built on (\autoref{sec:nns}), the basic ideas of language modeling (\autoref{sec:lm-basics}) and the way \acp{lm} are constructed, trained, and eventually applied in \ac{nlp} (\Cref{sec:plms,sec:llms}).

\subsection{Neural Networks}
\label{sec:nns}
First, we need to build a tool for learning patterns from data\footnote{Until we get to \ac{d2t} generation in \autoref{sec:d2t}, we will use the word ``\textit{data}'' only in its abstract sense, as in ``any inputs we can apply our algorithms to''. We will use the term ``structured data'' whenever it is necessary to make the distinction.}. This tool---which for us will be the \textbf{neural networks}---will later help us with learning patterns about language from large-scale textual data, and in turn also with generating the language.

Let's say our goal is to predict the real-number output $y \in \mathbb{R}$ for the given vector of real numbers $\mathbf{x} = (x_1, \ldots, x_n) \in \mathbb{R}^n$.
% \footnote{We will follow the convention that vectors are denoted with boldface letters ($\mathbf{x}$), and real numbers with plain letters ($x$).} 
Let's also assume that the $\mathbf{x} \rightarrow y$ mapping is not arbitrary (that would leave us with memorizing all the $(\mathbf{x},y)$ pairs), but follows some regularities and underlying patterns that can be learned. This assumption will be naturally satisfied if we consider $(\mathbf{x},y)$ to be representations of real-word data, e.g. documents and their labels.

For learning the underlying pattern between $\mathbf{x}$ and $y$, we will use mathematical models designed to capture the pattern in their parameters. The idea is that the models estimate the parameters from a limited set of examples called the \textit{training data}:  $\mathcal{D_{\text{train}}} = \{(\mathbf{x}_1, y_1), \ldots, (\mathbf{x}_{n}, y_{n})\}$, and use the learned parameters to predict the outputs on the \textit{test data}: $\mathcal{D_{\text{test}}} = \{(\mathbf{x}_{n+1}, y_{n+1}), \ldots, (\mathbf{x}_{m}, y_{m})\}$.

\paragraph{Perceptron Algorithm} One of the early mathematical models designed for predicting the outputs based on the inputs is the \emph{perceptron algorithm} \cite{rosenblatt1958perceptron}. In this case, we assume the output is a binary class label: $y \in \{-1, 1\}$. The algorithm learns the parameters $\textbf{w} = (w_1, \ldots, w_n) \in \mathbb{R}^n$ and $b \in \mathbb{R}$ describing a linear decision boundary separating the data points according to their class label. The algorithm proceeds as follows:


\begin{enumerate}
    \item The parameters $\textbf{w}$ and $b$ are initialized to small random values or zeros.
    \item For each training example $(\mathbf{x}_i, y_i)$, the algorithm updates the weights and bias to adjust their current estimate towards the ground truth target:
          \begin{align} \label{eq:perceptron1}
              \hat{y}_i  & = \text{sign}(\textbf{w} \cdot x_i + b)       \\
              \textbf{w} & = \textbf{w} + (y_i - \hat{y}_i) \textbf{x}_i \\
              b          & = b + y_i - \hat{y}_i
          \end{align}
    \item The step (2) is repeated until convergence.
\end{enumerate}

The perceptron algorithm is guaranteed to converge if there exists a hyperplane which separates the data belonging to one class from another. However, it cannot learn a non-linear decision boundary  \cite{novikoff1962convergence}.

\paragraph{Multi-layer Perceptron} To overcome the fact that the perceptron is limited to linear decision boundaries, we can use a \textbf{\ac{mlp}}. This mathematical model---also known as a feed-forward neural network---is able to approximate any bounded continuous function \cite{hornik1989multilayer}.

As the name suggests, \ac{mlp} uses multiple perceptron-like units called \textit{neurons}. Analogically to the perceptron (\autoref{eq:perceptron1}), each neuron computes its output $o$ using the rule:
\begin{align}
    o & = f(\mathbf{w} \cdot \mathbf{x} + b),
\end{align}
% \begin{itemize}
where $f$ is the \emph{activation function}. Instead of signum, \ac{mlp} uses differentiable non-linear functions. The most common activation functions nowadays are \acl{relu} (\acs{relu}; \citealp{nair2010rectified}) or its adapted version \acl{gelu} (\acs{gelu}; \citealp{hendrycks2016gaussian}).

For efficiency, the neurons are organized in layers, which enables formulating the \ac{mlp} computations in terms of matrix multiplication. The $i$-th layer of \ac{mlp} is parametrized with a matrix $\mathbf{W}_i \in \mathbb{R}^{n\times m}$ and a vector of biases $\mathbf{b}_i \in \mathbb{R}^{m}$, performing the following computation:
\begin{align}
    \mathbf{h}_i & = f(\mathbf{W}_i \cdot \mathbf{h}_{i-1} + \mathbf{b}_i).
\end{align}

During training, we aim to minimize the \textit{loss function} describing the gap between the model predictions and the ground truth output. Since all the computations in \ac{mlp} are differentiable, we can use the chain rule to compute how each parameter in the network contributes to the loss. To minimize the loss, we use the \emph{backpropagation} algorithm, updating the network parameters in the reverse order of layers. The magnitude of the updates is controlled by the learning rate parameter $\alpha$.

\paragraph{Recurrent Neural Networks} Unlike \acp{mlp}, where the size of the input $\mathbf{x}$ is fixed, \acp{rnn} allow us to process sequences of arbitrary length. In its vanilla form, the \ac{rnn} maintains a hidden state $\mathbf{h}$ which gets updated for each unit $\mathbf{x}_t$ in sequence using the matrices $\mathbf{W}_h$, $\mathbf{W}_t$, and the bias $\mathbf{b}$.
\begin{align}
    \mathbf{h}_t = f(\mathbf{W}_h \mathbf{h}_{t-1} + \mathbf{W}_x \mathbf{x}_t + \mathbf{b})
\end{align}
\acp{rnn} have various shortcomings, such as repeated updates of the hidden state, which make it difficult to update using gradient-based methods, or inherently sequential processing, making it difficult to paralellize the models. However, they will serve as a springboard towards the Transformer architecture described in \autoref{sec:lm-basics}.


\subsection{Language Modeling}
\label{sec:lm-basics}
We will now look into how can we use neural networks to represent and process text.


\paragraph{One-Hot Encoding} Until now, we have assumed that the input is a vector of real numbers. However, a text is a sequence of discrete units such as characters, words, or subwords. To convert these units---called \textit{tokens}---to numerical representation, we can define a \textit{vocabulary} $V$ which assigns an integer index $i \in \{0, 1, \ldots, |V|-1\}$ to each token.

The naive way to represent each token would be using its integer value. However, this would misleadingly suggest linear dependence between tokens. A better way is to use the index $i$ for constructing a \textit{one-hot} vector $\mathbf{x} \in \{0,1\}^{|V|} $ for each token:
\begin{align}
    \mathbf{x}_j = \begin{cases}
        1 & \text{if } i = j, \\
        0 & \text{otherwise}.
    \end{cases}
\end{align}

While this representation is sound, it is still not very helpful -- it does not capture anything about the semantics of individual tokens and requires that the network learns to represent each token independently.

\paragraph{Word Embeddings}

\paragraph{Neural \acp{lm}}

\paragraph{Encoder-Decoder Framework}

\paragraph{Transformer Architecture}
% \label{sec:transformer}
\subsection{Pretrained Language Models}
\label{sec:plms}

\paragraph{Encoder Models}

\paragraph{Encoder-Decoder Models}

\paragraph{Decoder Models}

\subsection{Large Language Models}
\label{sec:llms}
\section{Data-to-Text Generation}
\label{sec:d2t}
\subsection{Rule-based Approaches}
\label{sec:rule-d2t}
\subsection{Neural Approaches}
\label{sec:neural-d2t}
\subsection{Datasets}
\label{sec:datasets}
\subsection{Evaluation Metrics}
\label{sec:evaluation}