% %%%%%%%%%%%%%%%%%%%%%%%%%%%%%%%%%%%%%%%%%%%%%%%%%%%%%%%%%%%%%%%%%%%%%%%%%%%%
\chapter{Introduction}
\label{chap:intro}
Producing \emph{natural language} comes \emph{natural} primarily to us, humans.
The key to computers' versatility and efficiency---their ``language''---are data structures: arrays, lists, trees and graphs, tables and databases.
% As we use the computers to ground our decisions, it is crucial for us to undestand and interpret the data stored in these structures. 
We can scrutinize the data with appropriate tools---provided sufficient domain expertise and enough time---but this does not address the core of the problem: that to most people, reading structured data is like trying to decipher a foreign language. As the volume of data in our world grows, it is tempting to turn the question on its head: Can we instead teach the computers to describe the structured data in our natural language?

% Eventually, as we try to understand and interpret the data stored in these structures. Do we need to scrutinize these structures ourselves, or can we hope to teach the computers to translate the data into our language?

This question has been addressed since the dawn of computing. The first attempts at producing natural language date back to the audacious attempts of \emph{translating} between English and Russian in 1950's \cite{sheridan1955research} which stirred a lot of excitement, and lead to a belief that \emph{producing} English sentences with a computer and a set of rules is a simpler task. Although in 1960's, people slowly began to ponder on its difficulties---\citet{yngve1961random} notes even the first ten sentences of a children's book provide \emph{``surprisingly wide linguistic diversity''} for assembling appropriate grammar rules---the overall sentiment was that language generation will soon be solved. The seminal work of \citet{winograd1971procedures}, describing in 461 pages the SHRDLU system which manipulates blocks in an imaginary block world according to user instructions, only glosses over presenting the state of the world to the user:
\begin{pquotation}{\citealp[p.384]{winograd1971procedures}}
    [R]esponses can be made as complex and varied as we want, since they are created by the programmer, and the program only repeats them.
\end{pquotation}
In other words: If we can make the computers mechanically repeat whatever we say, what else is there to generating language?

Fast forward to the present, the research world is beaming with excitement again: neural \acp{lm} have a suprising ability of producing the long-sought \emph{complex} and \emph{varied} language \cite{radford2019language,brown2020language}. Similarly to other tasks in \ac{ai}---from object recognition \cite{papert1966summer} to self-driving cars \cite{autonomouscars}---the apparent ease of the task for humans has proven deceptive. In the end, ot took us 50 years to build tools for generating fluent language. To make actual progress, we had to shift our attention from linguistic theories and rule-based systems, re-defining our systems in terms of data-based approaches and generic learning algorithms.

In the previous decades, in which language was generated using rules and grammars, the goal of systems for automatic \ac{nlg}---which has meanwhile established itself as a standalone scientific discipline, with its journals, conferences, and stable base of researchers \cite{ACLanthologySIGGEN}---was rather pragmatic.
% taking structured data from a particular system  and presenting it to the users in the form they will understand. 
The works in \ac{nlg} were the works of \emph{engineering}: natural language was simply taken as one of the suitable mediums to present the structured data to the users in an understandable form. From chart captioning \cite{mittalDescribingComplexCharts1998} to generating graph descriptions \cite{sunDomainIndependentSentence2006}, from weather forecast systems \cite{belzAutomaticGenerationWeather2008} to healthcare report generators \cite{portetAutomaticGenerationTextual2009}, the research papers read like \emph{how-to's} for building robust systems with widely adopted tools. As a result, the \ac{nlg} systems from that time were accurate and reliable, if only a bit too domain-specific and rigid \cite{reiterBuildingAppliedNatural1997,gattSurveyStateArt2018}.

% The essence of these systems was transforming a non-linguistic inputs to linguistic outputs according to a sequence of rule. In some sense, we can therefore say that transforming data to text---what is now specifically called \ac{d2t} generation---was all there was to \ac{nlg}.

With neural models, the  \ac{nlg} as a research field has changed. Most notably, it has become more experimental. Although neural \acp{lm} opened up fascinating possibilities in building end-to-end systems and solving the long-standing issues with fluency and domain-independence \cite{ferreiraNeuralDatatotextGeneration2019,dusekEvaluatingStateoftheartEndtoEnd2020,sharmaInnovationsNeuralDatatotext2022}, working with neural models had turned out to be closer to behavioral sciences than engineering \cite{holtzmanGenerativeModelsComplex2023}. As the researchers began adapting to the change in the paradigm (still an ongoing endeavor), the issues with respect to experimental design and evaluation came to surface \cite{gehrmannRepairingCrackedFoundation2022}, and by some, the whole change was percieved as a step back \cite{reiter2020academic}. The shift towards experimental approaches has also created a gap between research and industry; the industry opting for established approaches meeting industrial standards
% in the ever-changing research landscape
\cite{daleNaturalLanguageGeneration2020,daleNavigatingTextGeneration2023}.


Nevertheless, this progressive approach turned out to have its merits. The emphasis on open research in \ac{nlp}---where publicly releasing papers, code, and models has become a \emph{de facto} standard---has allowed everybody to stand on the proverbial shoulders of giants. As anybody can build on other's code within minutes since it is made public, the research is accelerating, gathering more observations, and moving toward better approaches. The convergence towards generic aproaches has also lead to heavy cross-pollination of ideas, making ideas for specific tasks applicable to other tasks. As such, \ac{nlg} is helping to advance other areas of \ac{nlp} and contribute to general knowledge on natural language, its production and processing.

Finally, as we gained other ways to generate language than from structured data: summarize and paraphrase other texts, continue incomplete text segments, generate answers to questions, or describe images and videos \cite{Dong2021ASO}, the original field has gradually adopted the---perhaps more apt---name of \emph{\ac{d2t} generation}. As one of the \ac{nlg} subfields, it tries to solve the long-stading issues with language fluency or flexibility of the systems using neural models.

The thesis inevitably reflects the shifts in \ac{nlp} research between 2020 and 2024: from early attempts at generating fluent language with neural \acp{lm}, towards dealing with issues of domain generalization and semantic accuracy with larger models.

% On the way, we touch various facets of \ac{d2t} generation: improving \ac{d2t} generation itself (\autoref{chap:low-res}), evaluating generated texts (\autoref{chap:evaluation}), processing and visualizing data (\autoref{chap:tabgenie}), and investigating system behavior (\autoref{chap:investigating}).






\section{Motivation}
\label{sec:rq}

The main goal of the thesis is to reconcile the gap outlined in the introduction: turning experimental approaches into reliable and accurate \ac{d2t} generation systems. Along the way, we study how do the models behave, how do our data look like, and how to evaluate model outputs. As a premise, we take neural \acp{lm}\footnote{For brevity, we will use ``\acp{lm}'' to denote ``neural \acp{lm}'' throughout the work, unless stated otherwise.} as mean of producing fluent and natural-sounding text, and study how to integrate them in \ac{d2t} systems while adhering to strict demands on the semantic accuracy of the output.

To make the problem more tangible, we divided it into the following research questions:

\begin{description}
    \item[RQ1] \textbf{In which scenarios are \acp{lm} useful for \ac{d2t} generation?} Identifying the strong sides of \acp{lm} for \ac{d2t} generation is essential before we proceed with integrating them into \ac{d2t} generation systems.
    \item[RQ2] \textbf{How to efficiently process the structured data with \acp{lm}?} With the knowledge gathered during pre-training on large text corpora, \acp{lm} can produce texts across many domains even in low-resource settings. However, \acp{lm} were pre-trained on modeling \emph{plain text}, while the structured data have specific \emph{inner structure}. If we want to efficiently use the \acp{lm}, we need to find the way to transform the data into a suitable input format while keeping the structure and other meta-information intact.
    \item[RQ3] \textbf{Can we make \acp{lm}-based systems more controllable?} A neural component in the system raises issues with controllability. We seek to build systems which are more predictable and have intermediate outputs which can be subject to examination.
    \item[RQ4] \textbf{How to evaluate the outputs of \ac{d2t} generation systems?}
    \item[RQ5] \textbf{How do the neural \ac{d2t} generation systems behave?}
\end{description}



\section{Main Contributions}
\label{sec:contributions}


Our main contributions, corresponding to the research questions outlined above, are:
\begin{enumerate}
    \item We show that with a very \textbf{simple \ac{lm}-based baseline}, we can achieve strong results on a shared task of generating texts from a knowledge graph (\autoref{sec:finetuning}).
    \item We show how to \textbf{transform the data to intermediate text-like input} suitable for \acp{lm} using hand-crafted or automatically extracted templates (\Cref{sec:iterative,sec:pipeline,sec:sem-acc}), rule-based \ac{nlg} methods (\autoref{sec:eval-token}), and specialized \acp{lm} (\autoref{sec:describing}).
    \item We show how we can use \acp{lm} limited to improving text fluency for building \textbf{more controllable \ac{d2t} generation systems} with an iterative approach (\autoref{sec:iterative}) and modular architecture (\autoref{sec:pipeline}).
    \item We develop \textbf{\ac{lm}-based automatic metrics} for evaluating outputs of \ac{d2t} generation systems on the level of data item mentions (\autoref{sec:sem-acc}) and output tokens (\autoref{sec:eval-token}).
    \item We build a \textbf{tool for visualizing} the structured data and model outputs (\autoref{sec:tabgenie}), and we \textbf{study \ac{lm} behaviors} across multiple \ac{d2t} tasks, data formats, and domains (\Cref{sec:prompting,sec:describing}).
\end{enumerate}



\section{Thesis Overview}
\label{sec:overview}
% \subsection*{\autoref{chap:low-res}}

% \begin{itemize}
%     \item Finetuning LMs: Train Hard, Finetune Easy: Multilingual Denoising for RDF-to-Text Generation \cite{kasnerTrainHardFinetune2020}
% \end{itemize}

\begin{table*}[ht]
    \small
    \begin{tabular}{p{4cm}p{7.5cm}p{1cm}}
        \toprule
        \textbf{Publication}                             & \textbf{Author contribution} & \textbf{Sec.}         \\ \midrule
        \multicolumn{3}{l}{\textbf{\autoref{chap:low-res}: Low-Resource Data-to-Text Generation}}               \\
        \citet{kasnerTrainHardFinetune2020}              & TODO                         & §\ref{sec:finetuning} \\
        \citet{kasnerDatatoTextGenerationIterative2020}  & TODO                         & §\ref{sec:iterative}  \\
        \citet{kasner2022neural}                         & TODO                         & §\ref{sec:pipeline}   \\ \cdashlinelr{1-3}
        \multicolumn{3}{l}{\textbf{\autoref{chap:evaluation}: Evaluating Generated Text}}                       \\
        \citet{dusekEvaluatingSemanticAccuracy2020}      & TODO                         & §\ref{sec:sem-acc}    \\
        \citet{kasnerTextinContextTokenLevelError2021}   & TODO                         & §\ref{sec:eval-token} \\ \cdashlinelr{1-3}
        \multicolumn{3}{l}{\textbf{\autoref{chap:tabgenie}: Data Processing and Visualization}}                 \\
        \citet{kasnerTabGenieToolkitTabletoText2023}     & TODO                         & §\ref{sec:tabgenie}   \\ \cdashlinelr{1-3}
        \multicolumn{3}{l}{\textbf{\autoref{chap:investigating}: Investigating Model Capabilities}}             \\
        \citet{kasnerMindLabelsDescribing2022}           & TODO                         & §\ref{sec:describing} \\
        \citet{kasnerReferenceBasedMetricsAnalyzing2024} & TODO                         & §\ref{sec:prompting}  \\\bottomrule
    \end{tabular}
\end{table*}