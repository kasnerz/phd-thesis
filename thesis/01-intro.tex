% %%%%%%%%%%%%%%%%%%%%%%%%%%%%%%%%%%%%%%%%%%%%%%%%%%%%%%%%%%%%%%%%%%%%%%%%%%%%
\chapter{Introduction}%
\label{chap:intro}
% %%%%%%%%%%%%%%%%%%%%%%%%%%%%%%%%%%%%%%%%%%%%%%%%%%%%%%%%%%%%%%%%%%%%%%%%%%%%

Here goes the text. You can use acronyms, such as \ac{mt}. Plural acronyms can
be generated by \acp{rnn}. \Acl{nar} renders as capitalized (and long). Some
extra use-cases include introducing the acronym and the citation both at the
same time, like \emph{\acl{nmt}} (\acs{nmt}\glsunset{nmt};
\citealp{bahdanau2014neural,vaswani2017attention}).  Sometimes, you do not need
to introduce the long form, in that case just use \acs{bert}
\citep{devlin-etal-2019-bert} or \acsp{bleu} everywhere. And make sure you
downloaded fresh ACL anthology.
