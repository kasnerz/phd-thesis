Data-to-text generation systems need to produce texts with high levels of semantic accuracy. Rule-based systems can guarantee this aspect, but their fluency and adaptability to new domains remain limited. Meanwhile, neural language models can easily generate fluent texts and adapt to new domains but are notoriously prone to producing inaccurate outputs. In this thesis, we explore how to efficiently employ neural components in data-to-text generation systems to get the best of both worlds. We focus on approaches based on pretrained transformer language models. Primarily, the models serve us as building blocks for robust and data-efficient data-to-text generation systems. Along with that, we introduce novel model-based evaluation metrics, focusing on detecting errors in data-to-text outputs, and a toolkit for preprocessing and visualizing data-to-text generation datasets. We also analyze the behavior of pretrained and large language models in specific scenarios, including describing individual relations in knowledge graphs and generating texts from standard data formats. We conclude that while employing neural language models in data-to-text generation remains a delicate endeavor, neural components can improve the fluency of the output texts and make the systems adaptable to new domains and input formats. At the same time, the semantic accuracy of the outputs can remain high if the models are used for specific, well-defined subtasks for improving text quality. For future research, we emphasize the need for proper benchmarking with suitable evaluation metrics on real-world use cases.