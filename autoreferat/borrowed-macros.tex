%%% This file contains definitions of various useful macros and environments %%%
%%% Please add more macros here instead of cluttering other files with them. %%%

%%% Minor tweaks of style

% This macro defines a chapter, which is not numbered, but is included
% in the table of contents.
\def\chapwithtoc#1{
\chapter*{#1}
\addcontentsline{toc}{chapter}{#1}
}

\def\myglsxtrshort#1{
%\glsunset{#1}
\glsxtrshort{#1}
}

% Colors
\definecolor{blue}{rgb}{0.0, 0.5, 0.69}
\definecolor{yellow}{rgb}{0.94, 0.8, 0.0}
\definecolor{purple}{rgb}{0.62, 0.0, 0.77}
\definecolor{green}{rgb}{0.0, 1.0, 0.0}
%\definecolor{green}{rgb}{0.0, 0.66, 0.47}  % munsell
\definecolor{red}{rgb}{0.95, 0.0, 0.24}
\definecolor{orange}{rgb}{1.0, 0.5, 0.0}
\definecolor{lightblue}{rgb}{0.67, 0.9, 0.93}

\definecolor{darkolivegreen}{rgb}{0.33, 0.42, 0.18}
\definecolor{pink}{rgb}{1.0, 0.75, 0.8}

% Width Macros
\newcommand{\figfullwidth}{0.96\linewidth}
\newcommand{\fighalfwidth}{0.48\linewidth}
\newcommand{\figthirdwidth}{0.32\linewidth}

% Research question formatting
\def\rquestion#1#2{
    \bigskip
    \noindent
        \textbf{#1} \textit{#2}
    \bigskip
}
\def\rsubquestion#1#2{
    \bigskip
    \noindent
    \hangindent=2.5em
    \hangafter=1
        \textbf{#1} \textit{#2}
    \bigskip
}
\def\rquestionbody#1{
    \begin{adjustwidth}{2.5em}{}
    #1
    \end{adjustwidth}
    \bigskip
    %\hfill\begin{minipage}{\dimexpr\textwidth-1.5cm}
    %#1
    %\xdef\tpd{\the\prevdepth}
    %\end{minipage}
}

% Draw black "slugs" whenever a line overflows, so that we can spot it easily.
\overfullrule=1mm

%%% An environment for typesetting of program code and input/output
%%% of programs. (Requires the fancyvrb package -- fancy verbatim.)

%\DefineVerbatimEnvironment{code}{Verbatim}{fontsize=\small, frame=single}

%%% The field of all real and natural numbers
\newcommand{\R}{\mathbb{R}}
\newcommand{\N}{\mathbb{N}}
\newcommand{\Tau}{\mathrm{T}}

%%% Shortcuts
\newcommand{\bs}[1]{\boldsymbol{#1}}

%%% Circled numbers
\newcommand*\circled[1]{\tikz[baseline=(char.base)]{
            \node[shape=circle,draw,inner sep=2pt] (char) {#1};}}

%%% Useful operators for statistics and probability
\DeclareMathOperator{\pr}{\textsf{P}}
\DeclareMathOperator{\E}{\textsf{E}\,}
\DeclareMathOperator{\var}{\textrm{var}}
\DeclareMathOperator{\sd}{\textrm{sd}}

\DeclareMathOperator*{\argmin}{argmin}
\DeclareMathOperator*{\argmax}{argmax}
\DeclareMathOperator*{\softmax}{softmax}

\DeclareMathOperator{\relu}{ReLU}
\let\emptyset\varnothing{}

\DeclareMathOperator*{\sgn}{sgn}
\DeclareMathOperator{\attn}{Attn}
\DeclareMathOperator{\multihead}{Multihead}
\DeclareMathOperator{\concat}{concat}

\DeclarePairedDelimiter\ceil{\lceil}{\rceil}
\DeclarePairedDelimiter\floor{\lfloor}{\rfloor}

%%% Transposition of a vector/matrix
\newcommand{\T}[1]{#1^\top}

%%% Various math goodies
\newcommand{\goto}{\rightarrow}
\newcommand{\gotop}{\stackrel{P}{\longrightarrow}}
\newcommand{\maon}[1]{o(n^{#1})}
\newcommand{\abs}[1]{\left|{#1}\right|}
\newcommand{\dint}{\int_0^\tau\!\!\int_0^\tau}
\newcommand{\isqr}[1]{\frac{1}{\sqrt{#1}}}

%%% Various table goodies
\newcommand{\pulrad}[1]{\raisebox{1.5ex}[0pt]{#1}}
\newcommand{\mc}[1]{\multicolumn{1}{c}{#1}}
\newcommand{\mcl}[1]{\multicolumn{1}{l}{#1}}

% this is for the examples
\newcommand{\redund}[1]{\textcolor{black!30!red!100}{\underline{#1}}}
\newcommand{\greenund}[1]{\textcolor{black!40!green!100}{\underline{#1}}}
\newcommand{\blueund}[1]{\textcolor{black!30!blue!100}{\underline{#1}}}

% check and cross marks
\newcommand{\cmark}{\ding{51}}%
\newcommand{\xmark}{\ding{55}}%
